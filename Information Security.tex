\documentclass[10pt,a4paper,openany]{book}
%\documentclass[12pt,report,russian]{ncc}
%\usepackage{a4wide}
% Для векторых русских шрифтов в PDF не забудьте установить пакеты cm-super & cm-unicode
\usepackage{cmap}                       % Поддержка поиска русских слов в PDF (pdflatex)
\usepackage[X2, T2A]{fontenc}
%\usepackage[T2, OT1]{fontenc}
\usepackage[utf8]{inputenc}
\usepackage[english,german,italian,latin,russian]{babel}
\usepackage{indentfirst}                % Красная строка в первом абзаце
%\usepackage{misccorr}
%Может быть установлено 8pt, 9pt, 10pt, 11pt, 12pt, 14pt, 17pt, and 20pt
%\usepackage[12pt]{extsizes}
%\usepackage[mag=1000,a4paper,left=3cm,right=2cm,top=2cm,bottom=2cm,noheadfoot]{geometry}

% Подключение amsmath также даёт поддержку автоматических \dots
% см. http://tex.stackexchange.com/questions/77737/dots-versus-ldots-is-there-a-difference
% см. http://tex.stackexchange.com/questions/117730/what-is-the-difference-between-ldots-and-cdots
\usepackage{amsmath} % разрешить \texttt и аналогичные в формулах
\usepackage{amssymb} % дополнительные математические символы
\usepackage{graphicx} % поддержка изображений

%\usepackage{amsfonts, eucal, bm, color, }

\usepackage{algorithm, algorithmic}     % 'algorithm' environments
\floatname{algorithm}{Алгоритм}

\usepackage{arydshln}                   % dash lines in tables
\usepackage{caption}                    % titles for figures
\usepackage{enumerate}
\usepackage{enumitem}                   % кастомизация itemize/enumerate, напр. отказ от indent
\usepackage{fancybox}                   % страница в рамке
\usepackage{float}			% sub figures
\usepackage[totoc=true]{idxlayout}      % балансировка индексов на последней странице, индекс в ToC
\usepackage{lscape}                     % поддержка поворота страниц на 90 градусов для широких таблиц
\usepackage{makeidx}                    % index
\usepackage{multicol}                   % поддержка колонок
\setlength{\columnsep}{0.1cm}
\usepackage{multirow}                   % multirow cells in tables
\usepackage{subfig}			% sub figures
\usepackage{tikz}                       % векторная графика внутри TeX
\usepackage{tablefootnote}		% footnote в таблицах
\usepackage{wrapfig}			% sub figures
\usepackage{textcomp}                   % \No support

\usepackage[left=1.84cm, right=1.5cm, paperwidth=14cm, top=1.8cm, bottom=2cm, height=19.8cm, paperheight=20cm]{geometry}
\usepackage[parentracker=true,
  backend=biber,
  hyperref=auto,
  language=auto,
  citestyle=gost-numeric,
  defernumbers=true,
  bibstyle=gost-numeric,
  sortlocale=ru_RU
]{biblatex}								% библиография по ГОСТу
\addbibresource{bibliography.bib}

% поддержка гиперссылок; гиперссылки в pdf, должен быть последним загруженным пакетом
\ifx\pdfoutput\undefined
    \usepackage[unicode,dvips]{hyperref}
\else
    \usepackage[pdftex,colorlinks,unicode,bookmarks]{hyperref}
\fi

%\paperwidth=16.8cm \oddsidemargin=0cm \evensidemargin=0cm \hoffset=-0.33cm \textwidth=13.2cm
%\paperheight=24cm \voffset=-0.4cm \topmargin=0cm \headsep=0cm \headheight=0cm \textheight=19.8cm \footskip=0.9cm

% параметры PDF файла
\hypersetup{
    pdftitle={Защита информации},
    pdfauthor={Э. М. Габидулин, А. С. Кшевецкий, А. И. Колыбельников, С. М. Владимиров},
    pdfsubject=учебное пособие,
    pdfkeywords={защита информации, криптография, МФТИ}
}

% добавить точку после номера секции, раздела и~т.\,д.
\makeatletter
\def\@seccntformat#1{\csname the#1\endcsname.\quad}
\def\numberline#1{\hb@xt@\@tempdima{#1\if&#1&\else.\fi\hfil}}
\makeatother

% перенос слов с тире
%\lccode`\-=`\-
%\defaulthyphenchar=127

% изменить подписи к рисункам, таблицам и~т.\,д.
\captionsetup{labelsep=endash}          % Нумерационный заголовок и текст разделяются тире
\captionsetup{textformat=simple}        % Текст подписи будет напечатан как есть
%\captionsetup[table]{position=above}    % вертикальные отступы подписи таблицы для случая, когда подпись вверху
%\captionsetup[figure]{position=below}   % вертикальные отступы подписи рисунка для случая, когда подпись внизу

%% стиль главы и секции вверху страницы
%\pagestyle{fancy}
%%\renewcommand{\chaptermark}[1]{\markboth{#1}{}}
%\renewcommand{\sectionmark}[1]{\markright{#1}{}}
%
%%\fancyhf{}
%%\fancyfoot[СE,CO]{\thepage}
%%\fancyhead[LE]{\textsc{\nouppercase{\leftmark}}}
%\fancyhead[RO]{\textsc{\nouppercase{\rightmark}}}
%
%\fancypagestyle{plain}{ %
%\fancyhf{}                              % remove everything
%\renewcommand{\headrulewidth}{0pt}      % remove lines as well
%\renewcommand{\footrulewidth}{0pt}}

% запретить выходить за границы страницы
\sloppy

\newtheorem{theorem}{Теорема}[section]
\newtheorem{lemma}[theorem]{Лемма}
\newtheorem{definition}[theorem]{Определение}
\newtheorem{property}[theorem]{Утверждение}
\newtheorem{corollary}[theorem]{Следствие}
%\newtheorem{algorithm}[theorem]{Алгоритм}
\newtheorem{remark}[theorem]{Замечание}
\newcommand{\proof}{\noindent\textsc{Доказательство.\ }}

%\newtheorem{example}{\textsc{\textbf{Пример}}}
\newcommand{\example}{\textsc{\textbf{Пример.}} }
\newcommand{\exampleend}

\DeclareMathOperator{\ord}{ord}
\newcommand{\set}[1]{\mathbb{#1}}
\newcommand{\group}[1]{\mathbb{#1}}
\newcommand{\E}{\group{E}}
\newcommand{\F}{\group{F}}
\newcommand{\GF}[1]{\group{GF}(#1)}
\newcommand{\Gr}{\group{G}}
\newcommand{\R}{\group{R}}
\newcommand{\Z}{\group{Z}}
\newcommand{\MAC}{\textrm{MAC}}
\newcommand{\HMAC}{\textrm{HMAC}}
\newcommand{\PK}{\textrm{PK}}
\newcommand{\SK}{\textrm{SK}}

\newcommand{\langde}[1]{нем. \foreignlanguage{german}{\textit{#1}}}
\newcommand{\langen}[1]{англ. \foreignlanguage{english}{\textit{#1}}}
\newcommand{\langit}[1]{итал. \foreignlanguage{italian}{\textit{#1}}}
\newcommand{\langlat}[1]{лат. \foreignlanguage{latin}{\textit{#1}}}

% Русская типографика
\renewcommand\leq{\leqslant}
\renewcommand\geq{\geqslant}
\renewcommand\emptyset{\varnothing}
\renewcommand\kappa{\varkappa}
\renewcommand\epsilon{\varepsilon}
\renewcommand\phi{\varphi}
\newcommand*{\No}{\textnumero}

% Для раздела с задачами
\newcommand{\taskinit}{\newcounter{task-section}\setcounter{task-section}{0}\newcounter{task-number}}
\newcommand{\tasksection}{\addtocounter{task-section}{1}\setcounter{task-number}{0}}
\newcommand{\tasknumber}{\textbf{\No\addtocounter{task-number}{1}\arabic{task-section}.\arabic{task-number}.}~~}

%Наконец, существует способ дублировать знаки операций, который мы приведём безо всяких пояснений. Включив
%\newcommand*{\hm}[1]{#1\nobreak\discretionary{}{\hbox{\mathsurround=0pt #1}}{}}
%в преамбулу, можно написать $a\hm+b\hm+c\hm+d$, при этом в формуле a\hm+b\hm+c\hm+d при переносе знак + будет продублирован.

% Дублирование символов бинарных операций ("+", "-", "="), набранных в строчных формулах, при переносе на другую строку:
%%begin{latexonly}
%\renewcommand\ne{\mathchar"3236\mathchar"303D\nobreak
%      \discretionary{}{\usefont
%      {OMS}{cmsy}{m}{n}\char"36\usefont
%      {OT1}{cmr}{m}{n}\char"3D}{}}
%\begingroup
%\catcode`\+\active\gdef+{\mathchar8235\nobreak\discretionary{}%
% {\usefont{OT1}{cmr}{m}{n}\char43}{}}
%\catcode`\-\active\gdef-{\mathchar8704\nobreak\discretionary{}%
% {\usefont{OMS}{cmsy}{m}{n}\char0}{}}
%\catcode`\=\active\gdef={\mathchar12349\nobreak\discretionary{}%
% {\usefont{OT1}{cmr}{m}{n}\char61}{}}
%\endgroup
%\def\cdot{\mathchar8705\nobreak\discretionary{}%
% {\usefont{OMS}{cmsy}{m}{n}\char1}{}}
%\def\times{\mathchar8706\nobreak\discretionary{}%
% {\usefont{OMS}{cmsy}{m}{n}\char2}{}}
%\mathcode`\==32768
%\mathcode`\+=32768
%\mathcode`\-=32768
%%end{latexonly}

\makeindex

\begin{document}
\selectlanguage{russian}

%\layout

% рамка границ страницы http://www.ctan.org/tex-archive/macros/latex/contrib/fancybox/fancybox-doc.pdf
% сделать поиск по fancypage, thisfancypage
%\thisfancypage{}{\fbox}
%\thisfancypage{\fbox}{}
%\fancypage{}{\fbox}         % закомментировать
%\fancypage{\fbox}{\fbox}    % закомментировать
%\fancypage{\setlength{\fboxsep}{32pt}\fbox}{}

\title{Защита информации \\ Учебное пособие}
\author{Габидулин Эрнст Мухамедович \\ Кшевецкий Александр Сергеевич \\ Колыбельников Александр Иванович \\ Владимиров Сергей Михайлович}
\date{
 %   \textbf{\textsc{Черновой вариант. Может содержать ошибки.}} \\
%    \today
}
\maketitle
\setcounter{page}{3}

\newpage
%\thispagestyle{empty}
\setcounter{tocdepth}{2}
\tableofcontents
%\thispagestyle{empty}
\newpage

%\lhead[\leftmark]{}
%\rhead[]{\rightmark}

\input{foreword}

\input{Short_history_of_cryptography}

\input{definitions}

\chapter{Классические шифры}

В главе приведены наиболее известные \emph{классические} шифры, которыми можно было пользоваться до появления роторных машин. К ним относятся такие шифры, как: шифр Цезаря\index{шифр!Цезаря}, шифр Плейфера\index{шифр!Плейфера}, шифр Хилла\index{шифр!Хилла}, шифр Виженера\index{шифр!Виженера}. Они наглядно демонстрируют различные классы шифров.

\input{monoalphabetic_ciphers}

\input{bigrammnye_substitution_ciphers}

\input{hills_cipher}

% \subsection{Омофонные замены}
%
% Омофонными заменами называют криптопримитивы, в основе которых лежит замена групп символов открытого текста $M$ на группу символов $C$ с использованием ключа $K$. Такой метод шифрования вносит неоднозначность между $M$ и $C$, это позволяет защититься от методов частотного криптоанализа.
%  \subsection{шифрокоды}
%  Шифрокоды -- это класс шифров сочетающих в себе свойства кодов и помехозащищённости со свойствами шифра и обеспечения конфиденциальности.

\input{vigeneres_cipher}

\input{polyalphabetic_cipher_cryptanalysis}

\input{perfect_secure_systems}

\chapter{Блочные шифры}\label{chapter-block-ciphers}\index{шифр!блочный|(}

\input{block_ciphers}

\input{lucifer}

\input{Feistel_cipher}

\input{DES}

\input{GOST_28147-89}

\input{AES}

\input{GOST_R_34.12-2015}

\input{Block_cipher_modes}

\section{Некоторые свойства блочных шифров}

\input{feistel_network_reversibility}

\input{Feistel_cipher_without_s_blocks}

\input{Avalanche_effect}

\input{double_and_triple_ciphering}

\index{шифр!блочный|)}

\input{generators}

\input{stream-ciphers}

\chapter{Криптографические хэш-функции}\label{chapter-hash-functions}
\selectlanguage{russian}

Хэш-функции возникли как один из вариантов решения задачи <<поиска по словарю>>. Задача состояла в поиске в памяти компьютера (оперативной или постоянной) информации по известному ключу. Возможными способами решения были хранение, например, всего массива ключей (и указателей на содержимое) в отсортированном в некотором порядке, списке, либо в виде бинарного дерева. Однако наиболее производительным с точки зрения времени доступа (при этом обладая допустимой производительностью по времени модификации) стал метод хранения в виде хэш-таблиц. Этот метод ведёт своё происхождение из стен компании IBM (как и многое другое в программировании).

Метод хэш-таблиц подробно разобран в любой современной литературе по программированию~\cite{Knuth:2001:3}. Напомним лишь, что его идея состоит в разделении множества ключей по корзинам (bins) в зависимости от значения некоторой функции, вычисляемой по значению ключа. Причём функция подбирается таким образом, чтобы в разных корзинах оказалось одинаковое число (в идеале -- не более одного) ключей. При этом сама функция должна быть быстро вычислимой, а её значение должно легко конвертироваться в натуральное число, которое не превышает число корзин.

\emph{Хэш-функцией} (\langen{hash function}) называется отображение, переводящее аргумент произвольной длины в значение фиксированной длины.

\emph{Коллизией} хэш-функции называется пара значений аргумента, дающая одинаковый выход хэш-функции. Коллизии есть у любых хэш-функций, если количество различных значений аргумента превышает возможное количество значений результата функции (принцип ящиков Дирихле). А если не превышает, то и нет смысла использовать хэш-функцию.

\example
Приведём пример метода построения хэш-функции, называемого методом Меркла~---~Дамгарда\index{структура!Меркла~---~Дамгарда}~\cite{Merkle:1979, Merkle:1990, Damgard:1990}.

Пусть имеется файл $X$ в виде двоичной последовательности некоторой длины. Разделяем $X$ на несколько отрезков фиксированной длины, например по 256 символов:  $m_{1} ~\|~ m_{2} ~\|~ m_{3} ~\|~ \ldots ~\|~ m_{t}$. Если длина файла $X$ не является кратной 256 битам, то последний отрезок дополняем нулевыми символами и обозначаем $m'_{t}$.
Обозначим $t$ за новую длину последовательности. Считаем каждый отрезок $m_i, ~ i = 1, 2, \dots, t$ двоичным представлением целого числа.

Для построения хэш-функции используем рекуррентный способ вычисления. Предварительно введём вспомогательную функцию $\chi(m, H)$, называемую функцией компрессии или сжимающей функцией. Задаём начальное значение $H_{0} = 0^{256} \equiv \underbrace{000 \ldots 0}_{256} $. Далее вычисляем:
\[ \begin{array}{l}
    H_1 = \chi( m_1, H_0), \\
    H_2 = \chi( m_2, H_1), \\
    \dots,\\
    H_t = \chi( m'_t, H_{t-1}). \\
\end{array} \]
Считаем $H_{t} = h(X)$ хэш-функцией.
\exampleend

В программировании к свойствам хорошей хэш-функции относят:
\begin{itemize}
    \item быструю скорость работы,
    \item минимальное число коллизий.
\end{itemize}

Можно назвать и другие свойства, которые были бы полезны для хэш-функции в программировании. К ним можно отнести, например, отсутствие необходимости в дополнительной памяти (неиспользование <<кучи>>), простоту реализации, стабильность работы алгоритма (возврат одного и того же результата после перезапуска программы), соответствие результатов работы хэш-функции с результатами работы других функций, например, функций сравнения (см. например, описания функций \texttt{hashcode()}, \texttt{equals()} и \texttt{compare()} в языке программирования Java).

\textbf{Однонаправленной функцией}\index{функция!однонаправленная} $f(x)$ называется функция, обладающая следующими свойствами:
\begin{itemize}
    \item вычисление значения функции $f(x)$ для всех значений аргумента $x$ является \emph{вычислительно лёгкой} задачей;
    \item нахождение аргумента $x$, соответствующего значению функции $f(x)$, является \emph{вычислительно трудной} задачей.
\end{itemize}

Свойство однонаправленности, в частности, означает, что если в аргументе $x$ меняется хотя бы один символ, то для любого $x$ значение функции $H(x)$ меняется непредсказуемо.

\textbf{Криптографической хэш-функцией} $H(x)$ называется хэш-функция, имеющая следующие свойства:
\begin{itemize}
    \item однонаправленность: \emph{вычислительно невозможно} по значению функции найти прообраз;
    \item \emph{слабая устойчивость к коллизиям}\index{устойчивость к коллизиям} (слабо бесконфликтная функция): для заданного аргумента $x$ \emph{вычислительно невозможно} найти другой аргумент $y \neq x: ~ H(x) = H(y)$;
    \item \emph{сильная устойчивость к коллизиям} (cильно бесконфликтная функция): \emph{вычислительно невозможно} найти пару разных аргументов $x \neq y: ~ H(x) = H(y)$.
\end{itemize}

Из требования на устойчивость к коллизиям, в частности, следует свойство (близости к) равномерности распределения хэш-значений.

При произвольной длине последовательности $X$ длина хэш-функции $H(X)$ в российском стандарте ГОСТ Р 34.11-94 равна 256 символам, в американском стандарте SHA несколько различных значений длин: 160, 192, 256, 512 символов.

\section{ГОСТ Р 34.11-94}\index{хэш-функция!ГОСТ Р 34.11-94|(}
\selectlanguage{russian}

Представим описание устаревшего российского стандарта хэш-функции ГОСТ Р 34.11-94~\cite{GOST-94}.

Пусть $X$ -- последовательность длины 256 бит. Запишем $X$ тремя способами в виде конкатенации блоков:
\[ \begin{array}{ll}
    X & = X_4 ~\|~ X_3 ~\|~ X_2 ~\|~ X_1 = \\
    & = ~ \eta_{16} ~\|~ \eta_{15} ~\|~ \dots ~\|~ \eta_2 ~\|~ \eta_1 = \\
    & = ~ \xi_{32} ~\|~ \xi_{31} ~\|~ \dots ~\|~ \xi_2 ~\|~ \xi_1
\end{array} \]
с длинами 64, 16 и 8 бит соответственно.

Введём три функции:
\[ \begin{array}{ll}
    A(X) & \equiv A(X_1 ~\|~ X_2 ~\|~ X_3 ~\|~ X_4) = \\
        & = \left( X_1 \oplus X_2 \right) ~\|~ X_4 ~\|~ X_3 ~\|~ X_2, \\
    & \\
    \psi(X) & \equiv \psi(\eta_{16} ~\|~ \eta_{15} ~\|~ \dots ~\|~ \eta_2 ~\|~ \eta_1) = \\
        & = \left( \eta_1 \oplus \eta_2 \oplus \dots \oplus \eta_{15} \right) ~\|~
            \eta_{16} ~\|~ \eta_{15} ~\|~ \dots ~\|~ \eta_3 ~\|~ \eta_2, \\
    & \\
    P(X) & \equiv P(\xi_{32} ~\|~ \xi_{31} ~\|~ \dots ~\|~ \xi_2 ~\|~ \xi_1) = \\
        & = \xi_{\varphi(32)} ~\|~ \xi_{\varphi(31)} ~\|~ \dots ~\|~ \xi_{\varphi(2)} ~\|~ \xi_{\varphi(1)},
\end{array} \]
где $\varphi(s)$ -- перестановка байта, $s$ -- номер байта. Функции $A(X)$ и $\psi(X)$ -- регистры сдвига с линейной обратной связью.

Число $s$ однозначно представляется через целые числа $i,k$, и правило перестановки $\varphi(s)$ записывается:
\[ \begin{array}{c}
    s = i + 4 (k - 1) + 1, ~~ 0 \leq i \leq 3, ~ 1 \leq k \leq 8, \\
    \varphi(s) = 8 i + k. \\
\end{array} \]
Приведём пример. Пусть $s = 7$, тогда $i=2, k=2$. Находим перестановку $\varphi(7) = 8 \cdot 2 + 2 = 18$. Седьмой байт переместился на 18-ое место.

В российском стандарте функция компрессии двух 256-битовых блоков сообщения $M$ и результата хэширования предыдущего блока $H$ имеет вид
\[
    H' = \chi(M, H) = \psi^{61}(H \oplus \psi(M \oplus \psi^{12}(S))),
\]
где $\psi^j(X)$ -- суперпозиция $j$ функций $\psi( \psi( \dots ( \psi( X)) \dots ))$, ~ 256-битовый блок $S$ определяется ниже.

256-битовые блоки $H$ и $S$ представляются конкатенацией четырёх 64-битовых блоков
\[ \begin{array}{l}
    H = h_4 ~\|~ h_3 ~\|~ h_2 ~\|~ h_1, \\
    S = s_4 ~\|~ s_3 ~\|~ s_2 ~\|~ s_1, \\
    s_i = E_{K_i}( h_i), ~ i = 1, 2, 3, 4, \\
\end{array} \]
где $E_{K_i}( h_i)$ -- криптографическое преобразование 64-битового блока $h_i$ стандарта блочного шифрования ГОСТ 28147-89 с помощью ключа шифрования $K_i$.

Вычисление ключей $K_i$ производится через вспомогательные функции:
\[ \begin{array}{c}
    U_1 = H, ~~ V_1 = M, \\
    U_i = A(U_{i-1}) \oplus C_i, ~~ V_i = A(A(V_{i-1})), ~~ i = 2, 3, 4, \\
\end{array} \]
где $C_2, C_3, C_4$ -- 256-битовые блоки:
\[ \begin{array}{c}
    C_2 = C_4 = 0^{256}, \\
    C_3 = 1^8 0^8 1^{16} 0^{24} 1^{16} 0^8 (0^8 1^8)^2 1^8 0^8 (0^8 1^8)^4 (1^8 0^8)^4. \\
\end{array} \]
Окончательно получаем ключи
\[
    K_i = P(U_i \oplus V_i), ~ i = 1,2,3,4.
\]

%Ключевая хэш-функция определяется в виде $h_{K} (X)=h(KXK)$, где $h(X)$ -- некоторая стандартная хэш-функция, $K$ -- секретный ключ.
\index{хэш-функция!ГОСТ Р 34.11-94|)}


\section{Хэш-функция «Стрибог»}\label{section-stribog}\index{хэш-функция!«Стрибог»|(}
\selectlanguage{russian}

С 1 января 2013 года в России введён в действие новый стандарт на криптографическую хэш-функцию ГОСТ Р 34.11-2012~\cite{GOST-R:34.11-2012}. Неофициально новый алгоритм получил название <<Стрибог>>. При разработке хэш-функции авторы основывались на нескольких требованиях:

\begin{itemize}
	\item не должна быть уязвима к известным атакам;
	\item должна использовать хорошо изученные конструкции и преобразования;
	\item не должно быть лишних преобразований, каждое преобразование должно гарантировать выполнение определённых криптографических свойств;
	\item при наличии нескольких вариантов реализации требуемого свойства -- наиболее простой для анализа и реализации;
	\item максимальная производительность \emph{программной} реализации.
\end{itemize}

В соответствии с данными требованиями алгоритм новой хэш-функции основывается на хорошо изученных конструкциях Меркла~---~Дамгарда\index{структура!Меркла~---~Дамгарда}~\cite{Merkle:1979, Merkle:1990, Damgard:1990} и Миагучи~---~Пренеля\index{структура!Миагучи~---~Пренеля}~\cite{Espen:Mieghem:1989, Miyaguchi:Ohta:Iwata:1990:03, Miyaguchi:Ohta:Iwata:1990:11}, во внешней своей структуре практически полностью повторяя режим HAIFA\index{HAIFA} (\langen{HAsh Iterative FrAmework},~\cite{Biham:Dunkelman:2007}), использовавшийся в хэш-функциях SHAvite-3\index{хэш-функция|SHAvite-3} и BLAKE\index{хэш-функция|BLAKE}.

\begin{figure}[htb]
	\centering
	\includegraphics[width=0.95\textwidth]{pic/stribog-md}
  \caption{Использование структуры Меркла~---~Дамгарда в хэш-фукнции <<Стрибог>>}
  \label{fig:stribog-md}
\end{figure}

Как показано на рис.~\ref{fig:stribog-md}, входное сообщение разбивается на блоки по 512 бит (64 байта). Последний блок \emph{слева} дополняется последовательностью из нулей и  одной единицы до 512 бит (длина дополнения не учитывается в дальнейшем, когда длина сообщения используется как аргумент функций). Для каждой части сообщения вычисляется значение функции $g_N(h, m)$, которая в качестве аргумента использует текущий номер блока (умноженный на 512), результат вычисления для предыдущего блока и очередной блок сообщения. Также есть два завершающих преобразования. Первое вместо блока сообщения использует количество обработанных бит N (то есть длину сообщения), а второе -- арифметическую сумму значений всех блоков сообщения. В предположении, что функция $g_N(h, m)$ является надёжной для создания криптографически стойких хэш-функций, известно, что конструкция Меркла~---~Дамгарда позволяет получить хэш-функцию со следующими параметрами:

\begin{itemize}
	\item сложность построения прообраза: $2^n$ операций;
	\item сложность построения второго прообраза: $2^n / \left|M\right|$ операций;
	\item сложность построения коллизии: $2^{n/2}$ операций;
	\item сложность удлинения прообраза: $2^n$ операций.
\end{itemize}

Все параметры совпадают с аналогичными для идеальной хэш-функции, кроме сложности построения второго прообраза, который равен $2^n$ для идеального алгоритма.

В качестве функции $g_N(h, m)$ используется конструкция Миагучи~---~Пренеля\index{структура!Миагучи~---~Пренеля} (см. рис.~\ref{fig:stribog-mp}), которая является стойкой ко всем атакам, известным для схем однонаправленных хэш-функций на базе симметричных алгоритмов, в том числе к атаке с <<фиксированной точкой>>~\cite[стр. 502]{Schneier:2002}. \emph{Фиксированной точкой}\index{точка!фиксированная} называется пара чисел $(h, m)$, для которой у заданной функции $g$ выполняется $g(h, m) = h$.

\begin{figure}[htb]
	\centering
	\includegraphics[width=0.66\textwidth]{pic/stribog-mp}
  \caption{Использование структуры Миагучи~---~Пренеля в хэш-фукнции <<Стрибог>>}
  \label{fig:stribog-mp}
\end{figure}

\index{шифр!XSPL|(}В качестве блочного шифра используется новый XSPL-шифр, изображённый на рис.~\ref{fig:stribog-xspl}, отдельные элементы и идеи которого позже войдут в новый стандарт <<Кузнечик>>\index{шифр!<<Кузнечик>>} (см. раздел~\ref{section-grig}). Шифр является примером шифра на основе SP-сети\index{SP-сеть} (сети замен и перестановок), каждый раунд которого является набором обратимых преобразований над входным блоком.

\begin{figure}[htb]
	\centering
	\includegraphics[width=0.75\textwidth]{pic/stribog-xspl}
  \caption{XSPL-шифр в хэш-фукнции <<Стрибог>>}
  \label{fig:stribog-xspl}
\end{figure}

Каждый раунд XSPL-шифра, кроме последнего, состоит из следующих обратимых преобразований:
\begin{itemize}
	\item $X\left[C\right]$ -- побитовое сложение по модулю 2 с дополнительным аргументом $C$;
	\item $S$ -- нелинейная обратимая замена байтов;
	\item $P$ -- перестановка байтов внутри блока данных (транспонирование матрицы размером $8 \times 8$ из ячеек по одному байту каждая);
	\item $L$ -- обратимое линейное преобразование (умножение векторов на фиксированную матрицу).
\end{itemize}

Особенностью предложенного шифра является полная аналогия между алгоритмом развёртывания ключа и алгоритмом, собственно, преобразования открытого текста. В качестве <<раундовых ключей>> для алгоритма развёртывания ключа на первом раунде используется общее число уже обработанных бит хэш-функцией N, а на остальных раундах -- 512-битные константы, заданные в стандарте.\index{шифр!XSPL|)}

Новый алгоритм, согласно отдельным исследованиям, до полутора раз быстрее предыдущего стандарта ГОСТ Р 34.11-94\index{хэш-функция!ГОСТ Р 34.11-94}, используя 27 тактов на один байт входного сообщения (94~МБ/с), против 40 для старого стандарта (64~МБ/с)\footnote{Реализации тестировались на процессоре Intel Core i7-920 CPU @ 2.67~GHz и видеокарте NVIDIA GTX 580. См.~\cite{Lebedev:2013}}.

В 2014 году группа исследователей (\cite{Guo:Jean:Leurent:Peyrin:Wang:2014}) обнаружила недостаток в реализации конструкции HAIFA в хэш-фукнции <<Стрибог>>, который ведёт к уменьшению сложности атаки по поиску второго прообраза до $n \times 2^{n/2}$, то есть до $2^{266}$. Авторы работы получили первую премии в размере пятьсот тысяч рублей на конкурсе по исследованию хэш-функции «Стрибог», проводившимся Российским Техническим комитетом по стандартизации «Криптографическая защита информации» (ТК~26) при участии Академии криптографии Российской Федерации и при организационной и финансовой поддержке ОАО «ИнфоТеКС».

\index{хэш-функция!«Стрибог»|)}


\section{Имитовставка}\label{section-MAC}
\selectlanguage{russian}
\index{имитовставка}

Для обеспечения целостности и подтверждения авторства информации, передаваемой по каналу связи, используют \textbf{имитовставку} $\MAC$ (\langen{message authentication code}).

Имитовставкой называется \emph{криптографическая хэш-функция} $\MAC(K,m)$, зависящая от передаваемого сообщения $m$ и секретного ключа $K$ отправителя $A$, обладающая свойствами цифровой подписи:
\begin{itemize}
    \item получатель $B$, используя такой же или другой ключ, имеет возможность проверить целостность\index{целостность} и доказать принадлежность информации $A$;
    \item имитовставку невозможно фальсифицировать.
\end{itemize}

Имитовставка может быть построена либо на симметричной криптосистеме (в таком случае обе стороны имеют один общий секретный ключ), либо на криптосистеме с открытым ключом, в которой $A$ использует свой секретный ключ, а $B$ -- открытый ключ отправителя $A$.

Наиболее универсальный способ аутентификации сообщений через схемы ЭП на криптосистемах с открытым ключом состоит в том, что сторона $A$ отправляет стороне $B$ сообщение
    \[ m ~\|~ \textrm{ЭП}(K, h(m)), \]
где $h(m)$ -- криптографическая хэш-функция в схеме ЭП и $\|$ является операцией конкатенации битовых строк. Для аутентификации большого объёма информации этот способ не подходит из-за медленной операции вычисления подписи. Например, вычисление одной ЭП на криптосистемах с открытым ключом занимает порядка 10 мс на ПК. При средней длине IP-пакета 1 Кб, для каждого из которых требуется вычислить имитовставку, получим максимальную пропускную способность в $\frac{1 ~ \text{Kб}}{10 ~ \text{мс}} = 100$ Кб/с.

Поэтому для большого объёма данных, которые нужно аутентифицировать, $A$ и $B$ создают общий секретный ключ аутентификации $K$. Далее имитовставка вычисляется либо с помощью модификации блочного шифра, либо с помощью криптографической хэш-функции.

Для каждого пакета информации $m$ отправитель $A$ вычисляет $\MAC(K,m)$ и присоединяет его к сообщению $m$:
    \[ m ~ \|~ \MAC(K,m). \]
Зная секретный ключ $K$, получатель $B$ может удостовериться с помощью кода аутентификации, что информация не была изменена или фальсифицирована, а была создана отправителем.

Требования к длине кода аутентификации в общем случае такие же, как и для криптографической хэш-функции, то есть длина должна быть не менее 160-256 бит. На практике часто используют усечённые имитовставки.

Стандартные способы использования имитовставки сообщения следующие:
\begin{itemize}
    \item Если шифрование данных не применяется, отправитель $A$ для каждого пакета информации $m$ отсылает сообщение
        \[ m ~\|~ \MAC(K, m) .\]
    \item Если используется шифрование данных симметричной криптосистемой с помощью ключа $K_e$, то имитовставка с ключом $K_a$ может вычисляться как до, так и после шифрования:
        \[ E_{K_e}(m) ~\|~ \MAC(K_a, E_{K_e}(m)) ~~ \text{ или } ~~ E_{K_e}(m ~\|~ \MAC(K_a, m)). \]

\end{itemize}
Первый способ, используемый в IPsec\index{протокол!IPsec}, хорош тем, что для проверки целостности достаточно вычислить только имитовставку, тогда как во втором случае нужно дополнительно вначале расшифровать данные. С другой стороны, во втором способе, используемом в системе PGP\index{протокол!PGP}, защищённость имитовставки не зависит от потенциальной уязвимости алгоритма шифрования.

Вычисление имитовставки от пакета информации $m$ с использованием блочного шифра $E$ осуществляется в виде:
    \[ \MAC(K, m) = E_K(H(m)), \]
где $H$ -- криптографическая хэш-функция.

Имитовставка на основе хэш-функции обозначается $\HMAC$ (\langen{Hash-based MAC})\index{HMAC} и стандартно вычисляется в виде:
    \[ \HMAC(K, m) = H(K \| H(K \| m)). \]

Возможно также вычисление в виде:
    \[ \HMAC(K, m) = H(K \| m \| K). \]

В протоколе IPsec\index{протокол!IPsec} используется следующее вычисление кода аутентификации:
    \[ \HMAC(K, m) = H((K \oplus ~ \textrm{opad}) ~\|~ H((K \oplus ~ \textrm{ipad}) ~\|~ m)), \]
где $\textrm{opad}$ -- последовательность повторяющихся байтов
    \[ \text{\texttt{0x5C}}= [1011100]_2, \]
$\textrm{ipad}$ -- последовательность повторяющихся байтов
    \[ \text{\texttt{0x36}} = [00110110]_2, \]
которые инвертируют половину битов ключа. Считается, что использование различных значений ключа повышает криптостойкость.

В протоколе защищённой связи SSL/TLS\index{протокол!SSL/TLS}, используемом в интернете для инкапсуляции протокола HTTP\index{протокол!HTTP} в протокол SSL (HTTPS\index{протокол!HTTPS}), код $\HMAC$ определяется почти так же, как в IPsec. Отличие состоит в том, что вместо операции XOR для последовательностей $\textrm{ipad}$ и $\textrm{opad}$ осуществляется конкатенация:
    \[ \HMAC(K, m) = H((K ~\|~ \textrm{opad}) ~\|~ H((K ~\|~ \textrm{ipad}) ~\|~ m)). \]

Двойное хэширование\index{двойное хэширование} с ключом в
    \[ \HMAC(K, m) = H(K \| H(K \| m)) \]
применяется для защиты от атаки на расширение сообщений. Вычисление хэш-функции от сообщения $m$, состоящего из $n$ блоков $m_1 m_2 \dots m_n$, можно записать в виде:
    \[ H_i = f(H_{i-1}, m_i), ~ H_0 \equiv IV = \textrm{const}, ~ H(m) \equiv H_n, \]
где $f$ -- известная сжимающая функция.

Пусть имитовставка использует одинарное хэширование с ключом:
    \[ \MAC(K, m) = H(K \| m) = H (m_0 = K \| m_1 \| m_2 \| \dots \| m_n). \]
Тогда криптоаналитик, не зная секретного ключа, имеет возможность вычислить имитовставку для некоторого расширенного сообщения $m \| m_{n+1}$:
\[
    \MAC(K, m \| m_{n+1}) = \underbrace{H \left( K \| m_1 \| m_2 \| \dots \| m_n \right.}_{\MAC(K, m)} \left. \| m_{n+1} \right) =
\] \[
    f(\MAC(K, m), m_{n+1}).
\]


\section{Коллизии в хэш-функциях}

\subsection{Вероятность коллизии}
\selectlanguage{russian}

Если $k$-битовая криптографическая хэш-функция имеет равномерное распределение выходных хэш-значений по всем сообщениям, то, согласно парадоксу дней рождения\index{парадокс дней рождения} (см. раздел~\ref{section-birthday-padradox} в~приложении), среди
    \[ n_{1/2} \approx \sqrt{2 \ln 2} \cdot 2^{k/2} \]
случайных сообщений с вероятностью больше $1/2$ найдутся два сообщения с одинаковыми значениями хэш-функций, то есть произойдёт коллизия.

Криптографические хэш-функции должны быть равномерными по выходу, насколько это можно проверить, чтобы быть устойчивыми к коллизиям. Следовательно, для нахождения коллизии нужно взять группу из примерно $2^{k/2}$ сообщений.

Например, для нахождения коллизии в 96-битовой хэш-функции, которая, в частности, используется в имитовставке\index{имитовставка} $\MAC$ в протоколе IPsec\index{протокол!IPsec}, потребуется группа из $2^{48}$ сообщений, 3072 TB памяти для хранения группы и время на $2^{48}$ операций хэширования, что достижимо.

Если хэш-функция имеет неравномерное распределение, то размер группы с коллизией меньше чем $n_{1/2}$. Если для поиска коллизии достаточно взять группу с размером, много меньшим $n_{1/2}$, то хэш-функция не является устойчивой к коллизиям.

Например, для 128-битовой функции MD5\index{коллизия}\index{хэш-функция!MD5} Ван Сяоюнь (\langen{Xiaoyun Wang}) и Юй Хунбо (\langen{Hongbo Yu}) в 2005 г. представили атаку для нахождения коллизии за $2^{39} \ll 2^{64}$ операций~\cite{WangYu:2005}. Это означает, что MD5 взломана и более не может считаться надёжной криптографической хэш-функцией.


\input{hash-functions_combinations}


\chapter{Асимметричные криптосистемы}\label{chapter-public-key}
\selectlanguage{russian}

\emph{Асимметричной криптосистемой} или же \emph{криптосистемой с открытым ключом} (\langen{public-key cryptosystem, PKC}) называется криптографическое преобразование, использующее два ключа -- открытый и закрытый. Пара из \emph{закрытого}\index{ключ!закрытый} (\langen{private key, secret key, SK})\footnote{В контексте криптосистем с открытым ключом можно ещё встретить использование термина <<секретный ключ>>. Мы не рекомендуем использовать данный термин, чтобы не путать с секретным ключом\index{ключ!секретный}, используемым в симметричных криптосистемах.} и \emph{открытого}\index{ключ!открытый} (\langen{public key, PK}) ключей создаётся пользователем, который свой закрытый ключ держит в секрете, а открытый ключ делает общедоступным для всех пользователей. Криптографическое преобразование в одну сторону (шифрование) можно выполнить, зная только открытый ключ, а в другую (расшифрование) -- только зная закрытый ключ. Во многих криптосистемах из закрытого ключа теоретически можно вычислить открытый ключ, однако это является сложной вычислительной задачей.

Если прямое преобразование выполняется открытым ключом, а обратное -- закрытым, то криптосистема называется \emph{схемой шифрования с открытым ключом}. Все пользователи, зная открытый ключ получателя, могут зашифровать для него сообщение, которое может расшифровать только владелец закрытого ключа.

Если прямое преобразование выполняется закрытым ключом, а обратное -- открытым, то криптосистема называется \emph{схемой электронной подписи (ЭП)}. Владелец закрытого ключа может \emph{подписать} сообщение, а все пользователи, зная открытый ключ, могут проверить, что подпись была создана только владельцем закрытого ключа и никем другим.

Криптосистемы с открытым ключом снижают требования к каналам связи, которые требуются для передачи данных. В симметричных криптосистемах перед началом связи (перед шифрованием сообщения и его передачей) требуется передать или согласовать секретный ключ шифрования по защищённому каналу связи. Злоумышленник не должен иметь возможность ни прослушать данный канал связи, ни подменить передаваемую информацию (ключ). Для надёжной работы криптосистем с открытым ключом необходимо, чтобы злоумышленник не имел возможности подменить открытый ключ легального пользователя. Другими словами, криптосистема с открытым ключом, в случае использования открытых и незащищённых каналов связи, устойчива к пассивному криптоаналитику\index{криптоаналитик!пассивный}, но всё ещё должна предпринимать меры по защите от активного криптоаналитика\index{криптоаналитик!активный}.

Для предотвращения атак <<человек посередине>> (\langen{man-in-the-middle attack})\index{атака!<<человек посередине>>} с активным криптоаналитиком\index{криптоаналитик!активный}, который бы подменял открытый ключ получателя во время его передачи будущему отправителю сообщений, используют \emph{сертификаты открытых ключей}\index{сертификат открытого ключа}. Сертификат представляет собой информацию о соответствии открытого ключа и его владельца, подписанную электронной подписью третьего лица. В корпоративных информационных системах организация может обойтись одним лицом, подписывающим сертификаты. В этом случае его называют \emph{доверенным центром сертификации}, или \emph{удостоверяющим центром}. В глобальной сети Интернет для защиты распространения программного обеспечения (например, защиты от подделок в ПО) и проверок сертификатов в протоколах на базе SSL/TLS\index{протокол!SSL/TLS} используется иерархия удостоверяющих центров, рассмотренная в разделе~\ref{section-CAs}. При обмене личными сообщениями и при распространении программного обеспечения с открытым кодом вместо жёсткой иерархии может использоваться \emph{сеть доверия}\index{сеть доверия}. В сети доверия каждый участник может подписать сертификат любого другого участника. Предполагается, что подписывающий знает лично владельца сертификата и удостоверился в соответствии сертификата владельцу при личной встрече.

Криптосистемы с открытым ключом построены на основе односторонних (однонаправленных) функций c потайным входом. Под \emph{односторонней} функцией понимают \emph{вычислительную} невозможность вычисления её обращения: вычисление значения функции $y = f(x)$ при заданном аргументе $x$ является лёгкой задачей, вычисление аргумента $x$ при заданном значении функции $y$ -- трудной задачей.

Односторонняя функция $y = f(x,K)$ с \emph{потайным входом}\index{функция!с потайным входом} $K$ определяется как функция, которая легко вычисляется при заданном $x$, и аргумент $x$ которой можно легко вычислить из $y$, если известен <<секретный>> параметр $K$, и вычислить невозможно, если параметр $K$ неизвестен.

Примером подобной функции является возведение в степень по модулю составного числа $n$:
	\[ c = f \left( m \right) = m ^ e \mod n.\]

Для того, чтобы быстро вычислить обратную функцию
	\[ m = f^{-1} \left( c \right) = \sqrt[e]{c} \mod n, \]
её можно представить в виде
	\[ m = c^{d} \mod n,\]
где
	\[ d = e^{-1} \mod \varphi \left( n \right). \]

В последнем выражении $\varphi \left( n \right)$ -- это функция Эйлера\index{функция!Эйлера}. В качестве <<потайной дверцы>> или секрета можно рассматривать или непосредственно само число <<$d$>>, или значение $\varphi \left( n \right)$. Последнее можно быстро найти только в том случае, если известно разложение числа $n$ на простые сомножители. Именно эта функция с потайной дверцей лежит в основе криптосистемы RSA\index{криптосистема!RSA}.

Необходимые математические основы модульной арифметики, групп, полей и простых чисел приведены в приложении~\ref{chap:discrete-math}.

\section{Криптосистема RSA}\index{криптосистема!RSA|(}
\selectlanguage{russian}

\subsection{Шифрование}\index{шифр!RSA|(}

В 1978 г. Рональд Рив\'{е}ст, Ади Шамир и Леонард Адлеман (\langen{Ronald Linn Rivest, Adi Shamir, Leonard Max Adleman}, \cite{RSA:1978}) предложили алгоритм, обладающий рядом интересных для криптографии свойств. На его основе была построена первая система шифрования с открытым ключом, получившая название по первым буквам фамилий авторов -- система RSA.

Рассмотрим принцип построения криптосистемы шифрования RSA с открытым ключом.

\begin{enumerate}
    \item \textbf{Создание пары из закрытого и открытого ключей.}
        \begin{enumerate}
            \item Случайно выбрать большие простые\index{число!простое} различные числа $p$ и $q$, для которых $\log_2 p \simeq \log_2 q > 1024$ бит\footnote{Случайный выбор больших простых чисел не является простой задачей. См. раздел~\ref{section-pseudo-primes-generation} в приложении.}.
            \item Вычислить произведение $n = pq$.
            \item Вычислить функцию Эйлера\index{функция!Эйлера}\footnote{См. раздел~\ref{section-group-multiplicative} в приложении.} $\varphi(n) = (p-1)(q-1)$.
            \item Выбрать случайное целое число $e \in [3, \varphi(n)-1]$, взаимно простое с $\varphi(n)$: $~ \gcd(e, \varphi(n)) = 1$.
            \item Вычислить число $d$ такое, что $d \cdot e = 1 \mod \varphi(n)$.
            \item Закрытым ключом будем называть числа $n$ и $d$, открытым ключом -- $n$ и $e$\footnote{Некоторые авторы считают некорректным включать число $n$ в состав закрытого ключа, так как оно уже входит в открытый. Авторы настоящего пособия включают число $n$ в состав закрытого ключа, что в результате позволяет в дальнейшем использовать для расшифрования и создания электронной подписи данные \emph{только} из закрытого ключа, не прибегая к <<помощи>> данных из открытого ключа.}.
        \end{enumerate}

    \item \textbf{Шифрование с использованием открытого ключа}
        \begin{enumerate}
            \item Сообщение представляют целым числом $m \in [1, n-1]$.
            \item Шифротекст вычисляется как
                \[ c = m^e \mod n. \]
                Шифротекст -- также целое число из диапазона $[1, n-1]$.
        \end{enumerate}
    \item \textbf{Расшифрование с использованием закрытого ключа}

        Владелец закрытого ключа вычисляет
                \[ m = c^d \mod n. \]
\end{enumerate}

Покажем корректность схемы шифрования RSA. В результате расшифрования шифротекста $c$ (полученного путём шифрования открытого текста $m$) легальный пользователь имеет:
\[\begin{array}{ll}
    c^{d} & = m^{ed} \mod p = \\
          & = m^{ 1 + \alpha_1 \cdot \varphi(n)} \mod p = \\
          & = m^{ 1 + \alpha_1 \cdot ( p - 1 ) ( q - 1 )} \mod p = \\
          & = m^{ 1 + \alpha_2 \cdot ( p - 1 )} \mod p = \\
          & = m \cdot m^{\alpha_2 \cdot ( p - 1 )} \mod p. \\
\end{array}\]

Если $m$ и $p$ являются взаимно простыми, то из малой теоремы Ферма\index{теорема!Ферма малая} следует, что:
	\[m^{\left( p - 1 \right)} = 1 \mod p,\]
\[\begin{array}{ll}
	c^{d} & = m \cdot m^{\alpha_2 \cdot \left( p - 1 \right)} \mod p = \\
	      & = m \cdot \left( m^{\left(p - 1\right)} \right)^{\alpha_2} \mod p = \\
	      & = m \cdot 1^{\alpha_2} \mod p = \\
	      & = m \mod p.
\end{array}\]

Если же $m$ и $p$ не являются взаимно простыми, то есть $p$ является делителем $m$ (помним, что $p$ -- простое число), то $m = 0 \mod p$ и $c^{d} = 0 \mod p$.

В результате, для любых $m$ верно, что $c^{d} = m \mod p$. Аналогично доказывается, что $c^{d} = m \mod q$. Из китайской теоремы об остатках\index{теорема!китайская об остатках} (см. раздел~\ref{section-chinese-remainder-theorem} в приложении) следует:

\[\begin{cases}
	n = p \cdot q, \\
	c^{d} = m \mod p, \\
	c^{d} = m \mod q,
\end{cases} \Rightarrow c^{d} = m \mod n.\]

\example Создание ключей, шифрование и расшифрование в криптосистеме RSA.

\begin{enumerate}
    \item Генерирование параметров.
        \begin{enumerate}
            \item Выберем числа $p=13, q=11, n = 143$.
            \item Вычислим $\varphi(n) = (p-1)(q-1) = 12 \cdot 10 = 120$.
            \item Выберем $e=23: ~ \gcd(e, \varphi(n))=1, ~ e \in [3, 119]$.
            \item Найдём $d = e^{-1} \mod \varphi(n) = 23^{-1} \mod 120 = 47$.
            \item Открытый и закрытый ключи:
                \[ \PK = (e:23, n:143), ~ \SK = (d:47, n:143). \]
        \end{enumerate}
    \item Шифрование.
        \begin{enumerate}
            \item Пусть сообщение $m = 22 \in [1, n-1]$.
            \item Вычислим шифротекст:
                \[ c = m^e = 22^{23} \mod 143 = 55 \mod 143. \]
        \end{enumerate}
    \item Расшифрование.
        \begin{enumerate}
            \item Полученный шифротекст $c = 55$.
            \item Вычислим открытый текст:
                \[ m = c^d = 55^{47} \mod 143 = 22 \mod 143. \]
        \end{enumerate}
\end{enumerate}
\exampleend

%Рассмотрим её основные положения на примере криптосистемы с открытым ключом.
%Приведём общую схему алгоритма RSA.
%$C_i=M_{i}^{E_k}(mod N_j)$
%$N_j=P_{j}Q_{j}$
%$M_i=C_{i}^{D_k}(mod N_j)$
%$E_k\neq D_k$
%Вычислить $E_k$ из $D_k$ при длине блока сообщения $L_{блока} > L_{дополнения}$ можно только с экспоненциальной сложностью. $E_k D_K=1(mod \varphi(N_j))$
%Данное сравнение не даёт единственного решения. Решение данного сравнения и можно свести к следующему уравнению:
%$ax+by=1$
%$E_k D_k=k \varphi(N_j)+1$
%$1\leq E_k D_k <\varphi(N_j)$
%$\varphi(N_j)(-k)+ E_k D_k=1$
%Стандарт ISO X.509 определяет требования по реализации алгоритма RSA, в частности, требования к общесистемным параметрам и ключам, методы распространения сертификатов ключей и ключевых параметров, а также порядок ввода их в действие и многое другое.
\index{шифр!RSA|)}

\subsection{Электронная подпись}\index{электронная подпись!RSA|(}

Предположим, что пользователь $A$ не шифрует свои сообщения, но хочет посылать их в виде открытых текстов с подписью. Для этого надо создать электронную подпись (ЭП). Это можно сделать, используя систему RSA. При этом должны быть выполнены следующие требования:
\begin{itemize}
    \item вычисление подписи от сообщения является вычислительно лёгкой задачей;
    \item фальсификация подписи при неизвестном закрытом ключе -- вычислительно трудная задача;
    \item подпись должна быть проверяемой открытым ключом.
\end{itemize}

Создание параметров ЭП RSA производится так же, как и для схемы шифрования RSA. Пусть $A$ имеет закрытый ключ $\SK = (n, d)$, а получатель (проверяющий) $B$ -- открытый ключ $\PK = (e,n)$ пользователя $A$.

\begin{enumerate}
    \item $A$ вычисляет подпись сообщения $m \in [1,n-1]$ как
        \[ s = m^{d} \mod n \]
        на своём закрытом ключе $\SK$.
    \item $A$ посылает $B$ сообщение в виде $(m, s)$, где $m$ -- открытый текст, $s$ -- подпись.
    \item $B$ принимает сообщение $(m, s)$, возводит $s$ в степень $e$ по модулю $n$ ($e, n$ -- часть открытого ключа). В результате вычислений $B$ получает открытый текст
        \[ s^{d} \mod n = \left( m^{d} \mod n \right)^{e} \mod n = m. \]
    \item Сравнивает полученное значение с первой частью сообщения. При полном совпадении подпись принимается.
\end{enumerate}
Недостаток этой системы создания ЭП состоит в том, что подпись $m^{d} \mod n$ имеет большую длину, равную длине открытого сообщения $m$.

Для уменьшения длины подписи применяется другой вариант процедуры: вместо сообщения $m$ отправитель подписывает $h(m)$, где $h(x)$ -- известная криптографическая хэш-функция. Модифицированная процедура состоит в следующем.

\begin{enumerate}
    \item $A$ посылает $B$ сообщение в виде $(m, s)$, где $m$ -- открытый текст,
        \[ s = h(m)^d \mod n \]
        -- подпись.
    \item $B$ принимает сообщение $(m, s)$, вычисляет хэш $h(m)$ и возводит подпись в степень
        \[ h_1 = s^e \mod n. \]
    \item $B$ сравнивает значения $h(m)$ и $h_1$. При равенстве
        \[ h(m) = h_1 \]
        подпись считается подлинной, при неравенстве -- фальсифицированной.
\end{enumerate}


\example Создание и проверка электронной подписи в криптосистеме RSA.

\begin{enumerate}
    \item Генерирование параметров.
        \begin{enumerate}
            \item Выберем $p=13, q=17, n = 221$.
            \item Вычислим $\varphi(n) = (p-1)(q-1) = 12 \cdot 16 = 192$.
            \item Выберем $e=25: ~ \gcd(e = 25, \varphi(n) = 192) = 1, \\
                e \in [3, \varphi(n) - 1 = 191]$.
            \item Найдём $d = e^{-1} \mod \varphi(n) = 25^{-1} \mod 192 = 169$.
            \item Открытый и закрытый ключи:
                \[ \PK = (e:25, n:221), ~ \SK = (d:169, n:221). \]
        \end{enumerate}
    \item Подписание.
        \begin{enumerate}
            \item Пусть хэш сообщения $h(m) = 12 \in [1, n-1]$.
            \item Вычислим ЭП:
                \[ s = h^d = 12^{169} = 90 \mod 221. \]
        \end{enumerate}
    \item Проверка подписи.
        \begin{enumerate}
            \item Пусть хэш полученного сообщения $h(m) = 12$, полученная подпись $s = 90$.
            \item Выполним проверку:
                \[ h_1 = s^e = 90^{25} = 12 \mod 221, ~~ h_1 = h. \]
                Подпись верна.
        \end{enumerate}
\end{enumerate}

\index{электронная подпись!RSA|)}

\subsection{Семантическая безопасность шифров}

\textbf{Семантически безопасной}\index{криптосистема!семантически-безопасная} называется криптосистема, для которой вычислительно невозможно извлечь любую информацию из шифротекстов, кроме длины шифротекста. Алгоритм RSA не является семантически безопасным. Одинаковые сообщения шифруются одинаково, и следовательно применима атака на различение сообщений.

Кроме того, сообщения длиной менее $\frac{k}{3}$ бит, зашифрованные на малой экспоненте $e=3$, \emph{дешифруются} нелегальным пользователем извлечением обычного кубического корня.

В приложениях RSA используется только в сочетании с рандомизацией\index{рандомизация шифрования}. В стандарте PKCS\#1 RSA Laboratories описана схема рандомизации перед шифрованием OAEP-RSA (Optimal Asymmetric Encryption Padding). Примерная схема:
\begin{enumerate}
    \item Выбирается случайное $r$.
    \item Для открытого текста $m$ вычисляется
        \[ x = m \oplus H_1(r), ~ y = r \oplus H_2(x), \]
        где $H_1$ и $H_2$ -- криптографические хэш-функции.
    \item Сообщение $M = x ~\|~ y$ далее шифруется RSA.
\end{enumerate}
Восстановление $m$ из $M$ при расшифровании:
    \[ r = y \oplus H_2(x), ~ m = x \oplus H_1(r). \]

В модификации OAEP+ $x$ вычисляется как
    \[ x = (m \oplus H_1(r)) \| H_3(m \| r). \]

В описанной выше схеме ЭП под $m$ понимается хэш открытого текста, вместо шифрования выполняется подписание, вместо расшифрования -- проверка подписи.


\subsection{Выбор параметров и оптимизация}

\subsubsection{Выбор экспоненты $e$}

В случайно выбранной экспоненте $e$ c битовой длиной $k = \lceil \log_2 e \rceil$ одна половина битов в среднем равна 0, другая -- 1. При возведении в степень $m^e \mod n$ по методу <<возводи в квадрат и перемножай>> получится $k-1$ возведений в квадрат и в среднем
 $\frac{1}{2}(k-1)$ умножений.

Если выбрать $e$, содержащую малое число единиц в двоичной записи, то число умножений уменьшится до числа единиц в $e$.

Часто экспонента $e$ выбирается \emph{малым} \emph{простым} числом и/или содержащим малое число единиц в битовой записи для ускорения шифрования или проверки подписи, например:
\[
    \begin{array}{l}
        3 = [11]_2, \\
        17 = 2^4+1 = [10001]_2, \\
        257 = 2^8+1 = [100000001]_2, \\
        65537 = 2^{16}+1 = [10000000000000001]_2.
    \end{array}
\]

%Время шифрования или проверки подписи для малых экспонент становится $O(k^2)$ вместо $O(k^3)$, то есть в сотни раз быстрее для 1000-битовых чисел.


\subsubsection{Ускорение~шифрования по~китайской~теореме об~остатках}

Возводя $m$ в степень $e$ отдельно по $\mod p$ и $\mod q$ и применяя китайскую теорему об остатках\index{теорема!китайская об остатках} (Chinese remainder theorem, CRT), можно быстрее выполнить шифрование.

Однако ускорение шифрования в криптосистеме RSA через CRT может привести к уязвимостям в некоторых применениях, например в смарт-картах.

\example
Пусть $c = m^e \mod n$ передаётся на расшифрование на смарт-карту, где вычисляется
\[ \begin{array}{c}
    m_p = c^d \mod p, \\
    m_q = c^d \mod q, \\
    m = m_p q (q^{-1} \mod p) + m_q p (p^{-1} \mod q) \mod n. \\
\end{array} \]
Криптоаналитик внешним воздействием может вызвать сбой во время вычисления $m_p$ (или $m_q$), в результате получится $m_p'$ и $m'$ вместо $m$. Зная $m_p'$ и $m'$, криптоаналитик находит разложение числа $n$ на множители $p,q$:
    \[ \gcd(m' - m, ~ n) = \gcd( (m_p' - m) q (q^{-1} \mod p), ~ pq) = q. \]
\exampleend


\subsubsection{Длина ключей}

В 2005 году было разложено 663-битовое число вида RSA. Время разложения в эквиваленте составило 75 лет вычислений одного ПК. Самые быстрые алгоритмы факторизации -- субэкспоненциальные\index{задача!факторизации}. Минимальная рекомендуемая длина модуля $n$ = 1024 бита, но лучше использовать 2048 или 4096 бит.

В июле 2012 года NIST опубликовала отчёт~\cite{NIST:SP800:57}, который включал в себя таблицу сравнения надёжности ключей с разной длиной для криптосистем, относящихся к разным классам. Таблица была составлена согласно как известным на тот момент атакам на классы криптосистем, так и на конкретные шифры (см.~\ref{table:aesrsakeycompare}).
	
\begin{table}[h]
\begin{tabular}{|c|c|c|c|}
\hline
\multicolumn{1}{|p{0.2\linewidth}|}{бит безопасности} & \multicolumn{1}{|p{0.2\linewidth}|}{пример симметричного шифра} & \multicolumn{1}{|p{0.2\linewidth}|}{$\log_2 (n)$ для RSA\tablefootnote{Сравнимая по предоставляемой безопасности битовая длина произведения $n$ простых чисел $p$ и $q$ для криптосистем, основанных на сложности задачи факторизации числа $n$ на простые множители $p$ и $q$, в том числе RSA.}} & \multicolumn{1}{|p{0.2\linewidth}|}{$\log_2 (\| \group{G} \| )$ для эллиптических кривых\tablefootnote{Сравнимая по предоставляемой безопасности битовая длина количества элементов $\|\group{G}\|$ в выбранной циклической подгруппе $\group{G}$ группы точек $\group{E}$ эллиптической кривой для криптосистем, основанных на сложности дискретного логарифма в группах точек эллиптических кривых над конечными полями (см.~\ref{section-elliptic-curve-cryptosystems})}} \\
\hline
80	& 	2TDEA	&	1024	&	160--223	\\
112	& 	3TDEA	&	2048	&	224--255	\\
128	& 	AES-128	&	3072	&	256--383	\\
192	& 	AES-192	&	7680	&	384--511	\\
256	& 	AES-256	&	15360	&	512+	\\
\hline
\end{tabular}
\caption{Сравнимые длины ключей блочных симметричных шифров и ключевых параметров асимметричных шифров~\cite{NIST:SP800:57}}\label{table:aesrsakeycompare}
\end{table}

В приложении~\ref{section-modular-arithmetic} показано, что битовая сложность (количество битовых операций) вычисления произвольной степени $a^b \mod n$ является кубической $O(k^3)$, а возведения в квадрат $a^2 \mod n$ и умножения $a b \mod n$ -- квадратичной $O(k^2)$, где $k$ -- битовая длина чисел $a,b,n$.

%Увеличение длины модуля $n$ в 2 раза увеличивает время возведения в степень в $2^3$ раз для большой экспоненты $e$, а для маленькой экспоненты -- в $2^2$ раза.

\index{криптосистема!RSA|)}


\section{Криптосистема Эль-Гамаля}\index{криптосистема!Эль-Гамаля|(}
\selectlanguage{russian}

Эта система шифрования с открытым ключом опубликована в 1985 году Эль-Гамалем (Taher El Gamal, \cite{ElGamal:1985}). Рассмотрим принципы её построения.

Пусть имеется мультипликативная группа $\Z_p^* = \{1, 2, \dots, p-1\}$, где $p$ -- большое простое\index{число!простое} число, содержащее не менее 1024 двоичных разрядов. В группе $\Z_p^*$ существует $\varphi( \varphi( p ) ) = \varphi( p - 1 )$ элементов, которые порождают все элементы группы. Такие элементы называются генераторами.\footnote{Подробнее см. раздел~\ref{section-groups} в приложении.}

Выберем один из таких генераторов $g$ и целое число $x$ в интервале $1 \le x \le p-1$. Вычислим:
    \[ y = g^x \mod p. \]

Хотя элементы $x$ и $y$ группы $\Z_p^*$ задают друг друга однозначно, найти $y$, зная $x$, просто, а вот эффективного алгоритма для получения $x$ по заданному $y$ неизвестно. Говорят, что задача вычисления дискретного логарифма
	\[ x = \log_g y \mod p \]
является вычислительно сложной задачей. На сложности вычисления дискретного логарифма для больших простых $p$ основывается криптосистема Эль-Гамаля.

\subsection{Шифрование}\index{шифр!Эль-Гамаля|(}

Процедура шифрования в криптосистеме Эль-Гамаля состоит из следующих операций.

\begin{enumerate}
    \item \textbf{Создание пары из закрытого и открытого ключей стороной $A$.}
        \begin{enumerate}
            \item $A$ выбирает простое\index{число!простое} случайное число $p$.
            \item Выбирает генератор $g$ (в программных реализациях алгоритма генератор часто выбирается малым числом, например, $g = 2 \mod p$).
            \item Выбирает $x \in [2, p - 1]$ с помощью генератора случайных чисел.
            \item Вычисляет $y=g^{x}\mod p$.
            \item Создаёт закрытый и открытый ключи $\SK$ и $\PK$:
                \[ \SK = (p, g, x), ~ \PK = (p, g, y). \]
                Криптостойкость задаётся битовой длиной параметра $p$.
        \end{enumerate}
    \item \textbf{Шифрование на открытом ключе стороной $B$.}
        \begin{enumerate}
            \item Стороне $B$ известен открытый ключ $\text{PK} = (p, g, y)$ стороны $A$.
            \item Сообщение представляется числом $m \in [0, p-1]$.
            \item Выбирает случайное число $r \in [1, p-1]$ и вычисляет
                \[ \begin{array}{l}
                    a = g^r \mod p, \\
                    b = m \cdot y^r \mod p.
                \end{array} \]
            \item Создаёт шифрованное сообщение в виде
                \[ c = (a, b) \]
                и посылает стороне $A$.
        \end{enumerate}
    \item \textbf{Расшифрование на закрытом ключе стороной $A$.}

	Получив сообщение $(a, b)$ и владея закрытым ключом $\text{SK} = (p, g, x)$, $A$ вычисляет
                \[ m = \frac{b}{a^x} \mod p. \]
\end{enumerate}

Шифрование корректно, так как 
\[ \begin{array}{l}
    m' = \frac{b}{a^x} = \frac{m y^r}{g^{rx}} = m \mod p, \\
    m' \equiv m \mod p.
\end{array} \]

Чтобы криптоаналитику получить исходное сообщение $m$ из шифротекста $(a, b)$, зная только открытый ключ получателя $\text{PK} = (p, g, y)$, нужно вычислить значение $m = b \cdot y^{-r} \mod p$. Для этого криптоаналику нужно найти случайный параметр $r = \log_g a \mod p$, то есть вычислить дискретный логарифм. Такая задача является вычислительно сложной.

\example Создание ключей, шифрование и расшифрование в криптосистеме Эль-Гамаля.

\begin{enumerate}
    \item Генерация параметров.
        \begin{enumerate}
            \item Выберем $p=41$.
            \item Группа $\Z_p^*$ циклическая, найдём генератор (примитивный элемент). Порядок группы
                \[ |\Z_p^*| = \varphi(p) = p-1 = 40. \]
                Делители 40: 1, 2, 4, 5, 8, 10, 20. Элемент группы является примитивным, если все его степени, соответствующие делителям порядка группы, не сравнимы с 1. Из таблицы~\ref{tab:elgamal-generator-search} видно, что число $g = 6$ является генератором всей группы.
                \begin{table}[!ht]
                    \centering
                    \caption{Поиск генератора в циклической группе $\Z_{41}^*$. Элемент 6 -- генератор\label{tab:elgamal-generator-search}}
                    \resizebox{\textwidth}{!}{ \begin{tabular}{|c|c|c|c|c|c|c|c|c|}
                        \hline
                        \multirow{2}{*}{Элемент} & \multicolumn{7}{|c|}{Степени} & \multirow{2}{*}{Порядок элемента} \\
                        \cline{2-8}
                                & 2   & 4   & 5   & 8  & 10 & 20 & 40 & \\
                        \hline
                        2       & 4   & 16  & -9  & 10 & -1 & 1  &    & 20 \\
                        3       & 9   & -1  & -3  & 1  &    &    &    & 8 \\
                        5       & -16 & 10  & 9   & 18 & -1 & 1  &    & 20 \\
                        6       & -5  & -16 & -14 & 10 & -9 & -1 & 1  & 40 \\
                        \hline
                    \end{tabular} }
                \end{table}
            \item Выберем случайное $x = 19 \in [1, p-1]$.
            \item Вычислим
                \[ \begin{array}{ll}
                    y & = g^x \mod p = \\
                    & = 6^{19} \mod 41 = \\
                    & = 6^{1 + 2 + 4 \cdot 0 + 8 \cdot 0 + 16} \mod 41 = \\
                    & = 6^1 \cdot 6^2 \cdot 6^{4 \cdot 0} \cdot 6^{8 \cdot 0} \cdot 6^{16} \mod 41 = \\
                    & = 6 \cdot (-5) \cdot (-16)^0 \cdot 10^0 \cdot 18 \mod 41 = \\
                    & = -7 \mod 41.
                \end{array} \]
            \item Открытый и закрытый ключи:
                \[ \PK = (p:41, g:6, y:-7), ~ \SK = (p:41, g:6, x:19). \]
        \end{enumerate}
    \item Шифрование.
        \begin{enumerate}
            \item Пусть сообщением является число $m = 3 \in \Z_p^*$.
            \item Выберем случайное число $r = 25 \in [1, p-1]$.
            \item Вычислим
                \[ \begin{array}{l}
                    a = g^r \mod p = 6^{25} \mod 41 = 14 \mod 41, \\
                    b = m y^r \mod p = 3 \cdot (-7)^{25} \mod 41 = -9 \mod 41.
                \end{array} \]
            \item Шифротекстом является пара чисел
                \[ c = (a:14, ~ b:-9). \]
        \end{enumerate}
    \item Расшифрование.
        \begin{enumerate}
            \item Пусть получен шифротекст
                \[ c = (a:14, ~ b:-9). \]
            \item Вычислим открытый текст как
                \[ \begin{array}{ll}
                    m & = \frac{b}{a^x} \mod p = \\
                    & = -9 \cdot (14^{-1})^{19} \mod 41 = \\
                    & = -9 \cdot 3^{19} \mod 41 = \\
                    & = -9 \cdot (-14) \mod 41 = \\
                    & = 3 \mod 41. \\
                \end{array} \]
        \end{enumerate}
\end{enumerate}

\exampleend
\index{шифр!Эль-Гамаля|)}

\subsection{Электронная подпись}\index{электронная подпись!Эль-Гамаля|(}

Криптосистема Эль-Гамаля, как и криптосистема RSA\index{криптосистема!RSA}, может быть использована для создания электронной подписи.

По-прежнему имеются два пользователя $A$ и $B$ и незащищённый канал связи между ними. Пользователь $A$ хочет подписать своё открытое сообщение $m$ для того, чтобы пользователь $B$ мог убедиться, что именно $A$ подписал сообщение.

Пусть $A$ имеет закрытый ключ $\SK = (p, g, x)$, открытый ключ $\PK = (p, g, y)$ (полученные так же, как и в системе шифрования Эль-Гамаля) и хочет подписать открытое сообщение. Обозначим подпись $S(m)$.

Для создания подписи $S(m)$ пользователь $A$ выполняет следующие операции:
\begin{itemize}
    \item вычисляет значение криптографической хэш-функции $h(m) \in [0,p-2]$ от своего открытого сообщения $m$;
    \item выбирает случайное число $r, ~ \gcd(r, p-1)=1$;
    \item используя закрытый ключ, вычисляет
        \[ \begin{array}{l}
            a = g^r \mod p, \\
            b = \frac{h(m) - xa}{r} \mod (p-1); \\
        \end{array} \]
    \item создаёт подпись в виде двух чисел
        \[ S(m) = (a, b) \]
        и посылает сообщение с подписью $(m, S(m))$.
\end{itemize}

Получив сообщение, $B$ осуществляет проверку подписи, выполняя следующие операции:
\begin{itemize}
    \item по известному сообщению $m$ вычисляет значение хэш-функции $h(m)$;
    \item вычисляет
        \[ \begin{array}{l}
            f_1 = g^{h(m)} \mod p, \\
            f_2 = y^a a^b \mod p; \\
        \end{array} \]
    \item сравнивает значения $f_1$ и $f_2$, если
        \[ f_1 = f_2, \]
        то подпись подлинная, в противном случае -- фальсифицированная (или случайно испорченная).
\end{itemize}

Покажем, что проверка подписи корректна. По малой теореме Ферма получаем
\[ \begin{array}{ll}
    f_1 & = g^{h(m)} \mod p = \\
    & \\
    & = g^{h(m) \mod (p-1)} \mod p; \\
\end{array} \] \[ \begin{array}{ll}
    f_2 & = y^a a^b \mod p = \\
    & = \underbrace{\left( g^x \right)^a}_{y^a} \cdot
        \underbrace{\left( g^r \mod p \right)^{\frac{h(m) - xa}{r} \mod (p-1)}}_{a^b} \mod p = \\
    & \\
    & = g^{xa \mod (p-1)} ~\cdot~ g^{h(m) - xa \mod (p-1)} \mod p = \\
    & = g^{h(m) \mod (p-1)} \mod p = \\
    & = f_1.
\end{array} \]

\example Создание и валидация электронной подписи в криптосистеме Эль-Гамаля.

\begin{enumerate}
    \item Генерирование параметров.
        \begin{enumerate}
            \item Выберем $p=41$.
            \item Выберем генератор $g=6$ в группе $\Z_{41}^*$.
            \item Выберем случайное $x = 19 \in [1, p-1]$.%, ~ \gcd(x, p-1) = 1$.
            \item Вычислим
                \[ \begin{array}{ll}
                    y & = g^x \mod p = \\
                    & = 6^{19} \mod 41 = \\
                    & = 6^{1 + 2 + 4 \cdot 0 + 8 \cdot 0 + 16} \mod 41 = \\
                    & = 6 \cdot (-5) \cdot (-16)^0 \cdot 10^0 \cdot 18 \mod 41 = \\
                    & = -7 \mod 41. \\
                \end{array} \]
            \item Открытый и закрытый ключи:
                \[ \PK = (p:41, g:6, y:-7), ~ \SK = (p:41, g:6, x:19). \]
        \end{enumerate}
    \item Подписание.
        \begin{enumerate}
            \item От сообщения $m$ вычисляется хэш $h = H(m)$. Пусть хэш $h  = 3 \in [0, p-2]$.
            \item Выберем случайное $r = 9 \in [1, p-2]$: \\
                $\gcd(r=9, p-1 = 40) = 1$.
            \item Вычислим
                \[ \begin{array}{ll}
                    a & = g^r \mod p = \\
                      & = 6^9 \mod 41 = 19 \mod 41, \\
                    b & = \frac{h - xa}{r} \mod (p-1) = \\
                      & = (3 - 19 \cdot 19) \cdot 9^{-1} \mod 40 = \\
                      & = 2 \cdot 9 \mod 40 = 18 \mod 40. \\
                \end{array} \]
            \item Подпись
                \[ s = (a:19, b:18). \]
        \end{enumerate}
    \item Проверка подписи.
        \begin{enumerate}
            \item Для полученного сообщения находится хэш $h = H(m) = 3$. Пусть полученная подпись к нему имеет вид
                \[ s = (a:19, b:18). \]
            \item Вычислим
                \[ \begin{array}{ll}
                    f_1 & = g^h \mod p = \\
                        & = 6^3 \mod 41 = 11 \mod 41, \\
                    f_2 & = y^a a^b \mod p = \\
                        & = (-7)^{19} \cdot 19^{18} \mod 41 = 11 \mod 41. \\
                \end{array} \]
            \item Проверим равенство $f_1$ и $f_2$. Подпись верна, так как
                \[ f_1 = f_2 = 11. \]
        \end{enumerate}
\end{enumerate}

\exampleend
\index{электронная подпись!Эль-Гамаля|)}

\subsection{Криптостойкость}

Пусть дано уравнение $y=g^{x} \mod p$, требуется определить $x$ в интервале $0 < x < p-1$. Задача называется \textbf{дискретным логарифмированием}\index{задача!дискретного логарифмирования}.

Рассмотрим возможные способы нахождения неизвестного числа $x$. Начнём с перебора различных значений $x$ из интервала $0<x<p-1$ и проверки равенства $y=g^{x} \mod p$. Число попыток в среднем равно $\frac{p}{2}$, при $p=2^{1024}$ это число равно $2^{1023}$, что на практике неосуществимо.

Другой подход предложен советским математиком Гельфондом\index{алгоритм!Гельфонда} безотносительно к криптографии. Он состоит в следующем.
Вычислим $S=\left\lceil\sqrt{p-1}\right\rceil $, где скобки означают ближайшее (сверху) целое от числа $\sqrt{p-1} $.

Представим искомое число $x$ в следующем виде:

\begin{equation}
    x=x_{1} S+x_{2},
    \label{S}
\end{equation}

где $x_{1}$ и $x_{2}$ -- целые неотрицательные числа,
    \[ x_{1} \le S-1, ~ x_{2} \le S-1. \]

Такое представление является однозначным.

Вычислим и занесём в таблицу следующие $S$ чисел:
    \[ g^{0} \mod p, ~~ g^{1} \mod p, ~~ g^{2} \mod p, ~~ \dots, ~~ g^{S-1} \mod p. \]
Вычислим $g^{-S} \mod p$ и также занесём в таблицу.

\begin{center} \begin{tabular}{|l|c|c|c|c|c|c|}
    \hline
    $\lambda $ & 0 & 1 & 2 & \dots & $S-1$ & $-S$ \\
    \hline
    $g^{\lambda} \mod p$ & $g^{0}$ & $g^{1}$ & $g^{2}$ & \dots & $g^{S-1}$ & $g^{-S}$ \\
    \hline
\end{tabular} \end{center}

Для решения уравнения~\ref{S} используем перебор значений $x_{1}$.
\begin{enumerate}
    \item Предположим, что $x_{1} = 0$. Тогда $x = x_{2}$. Если число $y = g^{x_{2}} \mod p$ содержится в таблице, то находим его и выдаём результат: $x=x_{2} $. Задача решена. В противном случае переходим к пункту 2.
    \item Предположим, что $x_{1} =1$. Тогда $x=S+x_{2} $ и $y=g^{S+x_{2}} \mod p$. Вычисляем $yg^{-S} \mod p=g^{x_{2}} \mod p$. Задача сведена к предыдущей: если $g^{x_{2} } \mod p$ содержится в таблице, то в таблице находим число $x_{2} $ и выдаём результат $x$: $x=S+x_{2} $.
    \item Предположим, что $x_{1} =2$. Тогда $x=2S+x_{2} $ и $y=g^{2S+x_{2} } \mod p$. Если число $yg^{-2S} \mod p=g^{x_{2} } \mod p$ содержится в таблице, то находим число $x_{2}$ и выдаём результат: $x = 2S + x_{2}$.
     \item Пробегая все возможные значения, доберёмся, в худшем случае, до $x_{1} =S-1$. Тогда $x=(S-1)S+x_{2} $ и $y = g^{(S-1)S+x_{2} } \mod p$. Если число $yg^{-(S-1)S} \mod p=g^{x_{2}} \mod p$ содержится в таблице, то находим его и выдаём результат: $x=(S-1)S+x_{2}$.
\end{enumerate}

Легко проверить, что с помощью построенной таблицы мы проверили все возможные значения $x$. Максимальное число умножений равно $2S \approx 2\sqrt{p-1} =2\times 2^{512} $, что для практики очень велико. Тем самым проблему достаточной криптостойкости этой системы можно было бы считать решённой. Однако это неверно, так как числа $p-1$ являются составными. Если $p-1$ можно разложить на маленькие множители, то криптоаналитик может применить процедуру, подобную процедуре Гельфонда, по взаимно простым делителям $p-1$ и найти секрет. Пусть $p-1=st$. Тогда элемент $g^s$ образует подгруппу порядка $t$ и наоборот. Теперь, решая уравнение $y^s=(g^s)^a\mod p$, находим вычет $x=a\mod t$. Поступая аналогично, находим $x=b\mod s$ и по Китайской теореме об остатках находим $x$.

Несколько позже подобный метод ускоренного решения уравнения~\ref{S} был предложен Д. Шенксом (Daniel Shanks, \cite{Shanks:1971})\index{алгоритм!Шенкса}.

Пусть $k = \lceil \log_2 p \rceil$ -- битовая длина числа $p$. Алгоритм Гельфонда имеет экспоненциальную сложность (число двоичных операций):
    \[ O\left(\sqrt{p}\right) = O\left(e^{\frac{1}{2} \frac{1}{\log_2 e} k}\right). \]

Наилучшие из известных алгоритмов решения задачи дискретного логарифмирования имеют экспоненциальную сложность порядка
    \[ O\left(e^{\sqrt{k}}\right). \]

\index{криптосистема!Эль-Гамаля|)}


\input{elliptic_curve_cryptography}

\section{Длины ключей}
\selectlanguage{russian}

В таблице~\ref{tab:recommended-key-lengths} приведены битовые длины ключей для криптосистем.
%Традиционные рекомендации основаны на аппроксимации существующих алгоритмов для взлома на 10-30 лет вперед.

\begin{table}[!ht]
    \centering
    \caption{Минимальные длины ключей в битах по стандартам России и США\label{tab:recommended-key-lengths}}
    \resizebox{\textwidth}{!}{ \begin{tabular}{|l|c|c|c|c|}
        \hline
        & \multirow{2}{*}{\parbox{1.5cm}{Блочные шифры, $K$}} & \multicolumn{3}{|c|}{Схема ЭП} \\
        \cline{3-5}
        & & \parbox{1.5cm}{RSA\index{криптосистема!RSA}, $n$} & \parbox{2.3cm}{Эллипт. кривые, порядок точки} & \parbox{3.5cm}{Эль-Гамаль\index{криптосистема!Эль-Гамаля} $\mod p$: модуль / порядок (под)группы} \\
        \hline \hline
        \multicolumn{5}{|c|}{Взломано} \\
        \hline
        Биты & 56 & 663 & 109 & 503  \\
        Конкурс & \textsc{DesChal} & RSA-200 & ECC2K-108 &  \\
        Год & 1997 & 2005 & 2000 &  \\
        \hline \hline
        \multicolumn{5}{|c|}{Стандарт России} \\
        \hline
        Биты & 256 &  & 255 & \\
        ГОСТ & 28147—89 & --- & 34.10-2001 & --- \\
        Год & 1989 & & 2001 & \\
%       \hline
%       \multicolumn{2}{|l|}{\parbox{4cm}{Россия: нелицензируемая деятельность}} & \multicolumn{4}{c|}{40} \\
        \hline \hline
        \multicolumn{5}{|c|}{Стандарт США} \\
        \hline
        Биты & 128-256 & 1024-3072 & 151-480 & 1024-3072/160-256 \\
        FIPS \No & 197 & draft 186-3 & draft 186-3 & draft 186-3 \\
        Год & 2001 & 2006 & 2006 & 2006 \\
%       \hline
%       \multicolumn{2}{|l|}{\parbox{4cm}{США: экспортные ограничения до 2001 г.}} & 56 & 512 & 112 & 512/112 \\
%       \hline \hline
%       \multicolumn{2}{|l|}{Традиционные} & 80 & 1024 & 160 & 1024/160 \\
%       \cline{3-6}
%       \multicolumn{2}{|l|}{рекомендации} & 112 & 2048 & 224 & 2048/224 \\
%       \hline
%       \multicolumn{2}{|l|}{\parbox{4cm}{Рекомендация Lenstra, Verheul для 2010 г.}} & 78 & 1369 & 146-160 & 1369/138 \\
        \hline
    \end{tabular} }
\end{table}
%}\end{center}


\subsection*{Скорость вычисления ЭП}

Сравним производительность схем ЭП, чтобы продемонстрировать преимущества ЭП вида Эль-Гамаля\index{криптосистема!Эль-Гамаля} перед RSA\index{криптосистема!RSA} для больших ключей. В приложении показано, что в модульной арифметике по модулю числа $n$ с битовой длиной $k \simeq \log_2 n$ операции имеют битовую сложность:
\[ \begin{array}{lcl}
    a^b \mod n & - & O(k^3), \\
    ab \mod n, ~ a^{-1} \mod n & - & O(k^2), \\
    a+b \mod n & - & O(k). \\
\end{array} \]

Так как все описанные схемы ЭП используют возведение в степень по модулю, то битовая сложность -- $O(k^3)$. Оценки количества целочисленных $t$-разрядных умножений при вычислении ЭП имеют вид:
\begin{enumerate}
    \item RSA\index{электронная подпись!RSA}:
        \[ (2 \log_2 n) \cdot \left( \frac{\log_2 n}{t} \right)^2; \]
    \item DSA\index{электронная подпись!DSA} (\langen{Digital Signature Algorithm}, стандарт США~\cite{FIPS-PUB-186-4}), вычисляемая по принципу Эль-Гамаля\index{криптосистема!Эль-Гамаля} по модулю $p$ и с порядком циклической подгруппы $q$:
        \[ (2 \log_2 q) \cdot \left( \frac{\log_2 p}{t} \right)^2; \]
    \item ГОСТ Р 34.10-2001\index{электронная подпись!ГОСТ Р 34.10-2001} (стандарт России~\cite{GOST-2001}) и ECDSA\index{электронная подпись!ECDSA} (\langen{Elliptic Curve Digital Signature Algorithm}, стандарт США~\cite{FIPS-PUB-186-4}), вычисляемые по принципу Эль-Гамаля\index{криптосистема!Эль-Гамаля} в группе точек эллиптической кривой по модулю $p$:
        \[ (2 \log_2 p) \cdot 4 \cdot \left( \frac{\log_2 p}{t} \right)^2. \]
\end{enumerate}

В таблице~\ref{tab:signature-rate} приведены оценки скорости вычисления ЭП (оценки числа умножений 64-битовых слов).

\begin{table}[!ht]
    \centering
    \caption{Оценочное число 64-битовых умножений для вычисления ЭП\label{tab:signature-rate}}
    \begin{tabular}{|c|l|c|}
        \hline
        ЭП & Оценочное число 64-битовых умножений \\
        \hline \hline
        RSA\index{электронная подпись!RSA} 1024 & $(2 \cdot 1024) \cdot \left( \frac{1024}{64} \right)^2 \approx$ 500 000 \\
        RSA\index{электронная подпись!RSA} 2048 & 4 000 000 \\
        RSA\index{электронная подпись!RSA} 3072 & 14 000 000 \\
        RSA\index{электронная подпись!RSA} 4096 & 34 000 000 \\
        \hline \hline
        DSA\index{электронная подпись!DSA} 1024/160 & $(2 \cdot 160) \cdot \left( \frac{1024}{64} \right)^2 \approx$ 82 000 \\
        DSA\index{электронная подпись!DSA} 3072/256 & 1 200 000 \\
        \hline \hline
        ECDSA\index{электронная подпись!ECDSA} 160 & $(2 \cdot 160) \cdot 4 \cdot \left( \frac{160}{64} \right)^2 \approx$ 8 000 \\
        ECDSA\index{электронная подпись!ECDSA} 512 & 260 000 \\
        \hline \hline
        ГОСТ Р 34.10-2001\index{электронная подпись!ГОСТ Р 34.10-2001} & $(2 \cdot 256) \cdot 4 \cdot \left( \frac{256}{64} \right)^2 \approx$ 33 000 \\
        \hline
    \end{tabular}
\end{table}


\section{Инфраструктура открытых ключей}\label{chapter-public-key-infrastructure}

\subsection{Иерархия удостоверяющих центров}\label{section-CAs}
\selectlanguage{russian}

Проблему аутентификации и распределения сеансовых симметричных ключей шифрования в Интернете, а также в больших локальных и виртуальных сетях решают с помощью построения иерархии открытых ключей криптосистем с открытым ключом.

\begin{enumerate}
    \item Существует удостоверяющий центр (УЦ) верхнего уровня\index{удостоверяющий центр!верхнего уровня}, корневой УЦ\index{удостоверяющий центр!корневой} (\langen{root certificate authority, root CA}), обладающий парой из закрытого и открытого ключей. Открытый ключ УЦ верхнего уровня распространяется среди всех пользователей, причём все пользователи \emph{доверяют УЦ}. Это означает, что:
        \begin{itemize}
            \item УЦ -- <<хороший>>, то есть обеспечивает надёжное хранение закрытого ключа, не пытается фальсифицировать и скомпрометировать свои ключи;
            \item имеющийся у пользователей открытый ключ УЦ действительно принадлежит УЦ.
        \end{itemize}
        В массовых информационных и интернет-системах открытые ключи многих корневых УЦ встроены в дистрибутивы и пакеты обновлений ПО. Доверие пользователей неявно проявляется в их уверенности в том, что открытые ключи корневых УЦ, включённые в ПО, не фальсифицированы и не скомпрометированы. \emph{Де-факто пользователи доверяют а) распространителям ПО и обновлений, б) корневому УЦ.}\index{доверие}

        Назначение УЦ верхнего уровня -- проверка принадлежности и подписание открытых ключей других удостоверяющих центров второго уровня, а также организаций и сервисов. УЦ подписывает своим закрытым ключом следующее сообщение:
        \begin{itemize}
            \item название и URI УЦ нижележащего уровня или организации/сервиса,
            \item значение сгенерированного открытого ключа и название алгоритма соответствующей криптосистемы с открытым ключом,
            \item время выдачи и срок действия открытого ключа.
        \end{itemize}

    \item УЦ второго уровня\index{удостоверяющий центр} (\langen{certificate authority, certification authority, CA}) имеют свои пары открытых и закрытых ключей, сгенерированных и подписанных корневым УЦ, причём перед подписанием корневой УЦ убеждается в <<надёжности>> УЦ второго уровня, производит юридические проверки. Корневой УЦ не имеет доступа к закрытым ключам УЦ второго уровня.

        Пользователи, имея в своей базе открытых ключей доверенные открытые ключи корневого УЦ, могут проверить ЭП открытых ключей УЦ 2-го уровня и убедиться, что предъявленный открытый ключ действительно принадлежит данному УЦ. Таким образом:
        \begin{itemize}
            \item Пользователи полностью доверяют корневому УЦ и его открытому ключу, который у них хранится. Пользователи верят, что корневой УЦ не подписывает небезопасные ключи и гарантирует, что подписанные им ключи действительно принадлежат УЦ 2-го уровня.
            \item Проверив ЭП открытого ключа УЦ 2-го уровня с помощью доверенного открытого ключа УЦ 1-го уровня, пользователь верит, что открытый ключ УЦ 2-го уровня действительно принадлежит данному УЦ и не был скомпрометирован.
        \end{itemize}

        Аутентификация в протоколе защищённого интернет-соединения SSL/TLS\index{протокол!SSL/TLS} достигается в результате проверки пользователями совпадения URI-адреса сервера из ЭП с фактическим адресом.

        УЦ второго уровня в свою очередь тоже подписывает открытые ключи УЦ третьего уровня, а также организаций. И так далее по уровням.

    \item В результате построена \emph{иерархия} подписанных открытых ключей.

    \item Открытый ключ с идентификационной информацией (название организации, URI-адрес веб-ресурса, дата выдачи, срок действия и~др.) и подписью УЦ вышележащего уровня, заверяющей ключ и идентифицирующие реквизиты, называется \textbf{сертификатом открытого ключа},\index{сертификат открытого ключа} на который существует международный стандарт X.509\index{сертификат!X509}, последняя версия 3. В сертификате указывается его область применения: подписание других сертификатов, аутентификация для веба, аутентификация для электронной почты и~т.\,д.
\end{enumerate}

\begin{figure}[!ht]
	\centering
	\includegraphics[width=0.8\textwidth]{pic/X509-hierarchy}
	\caption{Иерархия сертификатов\label{fig:x509-hierarchy}}
\end{figure}

На рис.~\ref{fig:x509-hierarchy} приведены примеры иерархии сертификатов и путь подписания сертификата X.509\index{сертификат!X509} интернет-сервиса Google Mail.

Система распределения, хранения и управления сертификатами открытых ключей называется \textbf{инфраструктурой открытых ключей}\index{инфраструктура открытых ключей} (\langen{public key infrastructure, PKI}). PKI применяется для аутентификации в системах SSL/TLS\index{протокол!SSL/TLS}, IPsec\index{протокол!IPsec}, PGP и~т.\,д. Помимо процедур выдачи и распределения открытых ключей, PKI также определяет процедуру отзыва скомпрометированных или устаревших сертификатов.


\input{x509}


\input{key_distribution_protocols}

\input{secret-sharing}

\chapter{Примеры систем защиты}

\section{Система Kerberos для локальной сети}
\selectlanguage{russian}

Система аутентификации и распределения ключей Kerberos основана на протоколе Нидхема~---~Шрёдера. Самые известные реализации протокола Kerberos включают Microsoft Active Directory и ПО Kerberos с открытым кодом для Unix.

Протокол предназначен для решения задачи аутентификации и распределения ключей в рамках локальной сети, в которой есть группа пользователей, имеющих доступ к набору сервисов, для которых требуется обеспечить единую аутентификацию. Протокол Kerberos использует только симметричное шифрование. Секретный ключ используется для взаимной аутентификации.

Естественно, что в глобальной сети Интернет невозможно секретно создать и распределить пары секретных ключей, поэтому Kerberos построен для (виртуальной) локальной сети.

В протоколе используются 4 типа субъектов:

\begin{itemize}
    \item пользователи системы $C_i$;
    \item сервисы $S_i$, доступ к которым имеют пользователи;
    \item сервер аутентификации AS (Authentication Server), который производит аутентификацию пользователей по паролям и/или смарт-картам только один раз и выдаёт секретные сеансовые ключи для дальнейшей аутентификации;
    \item сервер выдачи мандатов TGS (Ticket Granting Server) для аутентификации доступа к запрашиваемым сервисам, аутентификация выполняется по сеансовым ключам\index{ключ!сеансовый}, выданным сервером AS.
\end{itemize}

Для работы протокола требуется заранее распределить следующие секретные симметричные ключи для взаимной аутентификации:
\begin{itemize}
    \item Ключи $K_{C_i}$ между пользователем $i$ и сервером AS. Как правило, ключом служит обычный пароль\index{пароль}, точнее, результат хэширования пароля. Может быть использована и смарт-карта.
    \item Ключ $K_{TGS}$ между серверами AS и TGS.
    \item Ключи $K_{S_i}$ между сервисами $S_i$ и сервером TGS.
\end{itemize}

\begin{figure}[!ht]
	\centering
	\includegraphics[width=\textwidth]{pic/kerberos}
	\caption{Схема аутентификации и распределения ключей Kerberos\label{fig:kerberos}}
\end{figure}

На рис.~\ref{fig:kerberos} представлена схема протокола, состоящая из 6 шагов.

Введём обозначения для протокола между пользователем $C$ с ключом $K_C$ и сервисом $S$ с ключом $K_S$:
\begin{itemize}
    \item $ID_C,\, ID_{TGS}, ID_S$ -- идентификаторы пользователя, сервера TGS и сервиса $S$ соответственно;
    \item $t_i,\, \tilde{t}_i$ -- запрашиваемые и выданные границы времени действия сеансовых ключей аутентификации;
    \item $ts_i$ -- метка текущего времени (timestamp);
    \item $N_i$ -- одноразовая метка (nonce)\index{одноразовая метка}, псевдослучайное число для защиты от атак воспроизведения сообщений;
    \item $K_{C,TGS},\, K_{C,S}$ -- выданные сеансовые ключи аутентификации пользователя и сервера TGS, пользователя и сервиса $S$ соответственно;
    \item $T_{TGS} = E_{K_{TGS}}(K_{C,TGS} ~\|~ ID_C ~\|~ \tilde{t}_1)$ -- мандат (ticket) для TGS, который пользователь расшифровать не может;
    \item $T_{S} = E_{K_S}(K_{C,S} ~\|~ ID_C ~\|~ \tilde{t}_2)$ -- мандат для сервиса $S$, который пользователь расшифровать не может;
    \item $K_1,\, K_2$ -- обмен информацией для генерирования общего секретного симметричного ключа дальнейшей коммуникации, например, по протоколу Диффи~---~Хеллмана\index{протокол!Диффи~---~Хеллмана}.
\end{itemize}

Схема протокола следующая:
\begin{enumerate}
    \item Первичная аутентификация пользователя по паролю, получение сеансового ключа $K_{C,TGS}$ для дальнейшей аутентификации. Это действие выполняется один раз для каждого пользователя, чтобы уменьшить риск компрометации пароля.
        \begin{enumerate}
            \item $C \rightarrow AS: ~~ ID_C ~\|~ ID_{TGS} ~\|~ t_1 ~\|~ N_1$.
            \item $C \leftarrow AS: ~~ ID_C ~\|~ T_{TGS} ~\|~ E_{K_C}( K_{C,TGS} ~\|~ \tilde{t}_1 ~\|~ N_1 ~\|~ ID_{TGS})$.
        \end{enumerate}
    \item Аутентификация сеансовым ключом $K_{C,TGS}$ на сервере TGS для запроса доступа к сервису выполняется один раз для каждого сервиса. Получение другого сеансового ключа аутентификации $K_{C,S}$.
        \begin{enumerate}
            \item $C \rightarrow TGS: ~~ ID_S ~\|~ t_2 ~\|~ N_2 ~\|~ T_{TGS} ~\|~ E_{K_{C,TGS}}(ID_C ~\|~ ts_1)$.
            \item $C \leftarrow TGS: ~~ ID_C ~\|~ T_{S} ~\|~ E_{K_{C,TGS}}( K_{C,S} ~\|~ \tilde{t}_2 ~\|~ N_2 ~\|~ ID_S)$.
        \end{enumerate}
    \item Аутентификация сеансовым ключом $K_{C,S}$ на сервисе $S$ -- создание общего сеансового ключа дальнейшего взаимодействия.
        \begin{enumerate}
            \item $C \rightarrow S: ~~ T_{S} ~\|~ E_{K_{C,S}}(ID_C ~\|~ ts_2 ~\|~ K_1)$.
            \item $C \leftarrow S: ~~ E_{K_{C,S}}( ts_2 ~\|~ K_2)$.
        \end{enumerate}
\end{enumerate}

Аутентификация и проверка целостности достигаются сравнением идентификаторов, одноразовых меток и меток времени внутри зашифрованных сообщений после расшифрования с их действительными значениями.

Некоторым недостатком схемы является необходимость синхронизации часов между субъектами сети.


\section{Pretty Good Privacy}
\selectlanguage{russian}

В качестве примера передачи файлов по сети с обеспечением аутентификации, конфиденциальности и целостности рассмотрим систему PGP (\langen{Pretty Good Privacy}), разработанную Филом Циммерманном (\langen{Phil Zimmermann}) в 1991 г. Изначально система предлагалась к использованию для защищённой передачи электронной почты. Стандартом PGP является OpenPGP. Примерами реализации стандарта OpenPGP являются GNU Privacy Guard (GPG) и netpgp, разработанные в рамках проектов GNU и NetBSD соответственно.

Каждый пользователь обладает одним или несколькими закрытыми ключами для криптосистемы с открытым ключом, которые используются для аутентификации посредством ЭП. Пользователь хранит также открытые ключи других пользователей, которые он использует для шифрования секретного сеансового ключа блочного шифрования. Передаваемое сообщение подписывается секретным ключом отправителя, затем сообщение шифруется блочной криптосистемой на случайно выбранном сеансовом ключе. Сам сеансовый ключ шифруется криптосистемой с открытым ключом на открытом ключе получателя.

Свои закрытые ключи отправитель хранит в зашифрованном виде. Набор ключей называется связкой закрытых ключей. Шифрование закрытых ключей в связке производится симметричным шифром\index{шифр!симметричный}, ключом которого является функция от пароля, вводимого пользователем. Шифрование закрытых ключей, хранимых на компьютере, является стандартной практикой для защиты от утечки, например, в случае взлома ОС, утери ПК и~т.\,д.

Набор открытых ключей других пользователей называется связкой открытых ключей.

\begin{figure}[!ht]
	\centering
	\includegraphics[width=0.9\textwidth]{pic/pgp}
	\caption{Схема обработки сообщения в PGP\label{fig:pgp}}
\end{figure}

На рис.~\ref{fig:pgp} представлена схема обработки сообщения в PGP для передачи от $A$ к $B$. Использование аутентификации, сжатия и блочного шифрования является опциональным. Обозначения на рисунке следующие.
\begin{itemize}
    \item Пароль -- пароль, вводимый отправителем для расшифрования связки своих закрытых ключей.
    \item $D$ -- расшифрование блочной криптосистемы для извлечения секретного ключа ЭП отправителя.
    \item $SK_A$ -- закрытый ключ ЭП отправителя.
    \item $ID_{SKa}$ -- идентификатор ключа ЭП отправителя, по которому получатель определяет, какой ключ из связки открытых ключей использовать для проверки подписи.
    \item $m$ -- сообщение (файл) для передачи.
    \item $h(m)$ -- криптографическая хэш-функция.
    \item $E_{SKa}$ -- схема ЭП на секретном ключе $SK_A$.
    \item $\|$ -- конкатенация битовых строк.
    \item $Z$ -- сжатие сообщения алгоритмом компрессии.
    \item $RND$ -- криптографический генератор псевдослучайной последовательности.
    \item $K_s$ -- сгенерированный псевдослучайный сеансовый ключ.
    \item $E_{Ks}$ -- блочное шифрование на секретном сеансовом ключе $K_s$.
    \item $PK_B$ -- открытый ключ получателя.
    \item $ID_{PKb}$ -- идентификатор открытого ключа получателя, по которому получатель определяет, какой ключ из связки закрытых ключей использовать для расшифрования сеансового ключа.
    \item $E_{PKb}$ -- шифрование сеансового ключа криптосистемой с открытым ключом на открытом ключе $B$.
    \item $c$ -- зашифрованное подписанное сообщение.
\end{itemize}


\section{Протокол SSL/TLS}\index{протокол!SSL/TLS}
\selectlanguage{russian}

Протокол SSL (\langen{Secure Sockets Layer}) был разработан компанией Netscape. Начиная с версии 3 протокол развивается как открытый стандарт TLS (\langen{Transport Layer Security}). Протокол SSL/TLS обеспечивает защищённое соединение по незащищённому каналу связи на прикладном уровне модели TCP/IP. Протокол встраивается между прикладным и транспортным уровнями стека протоколов TCP/IP. Для обозначения <<новых>> протоколов, полученных с помощью инкапсуляции прикладного уровня (HTTP\index{протокол!HTTP}, FTP\index{протокол!FTP}, SMTP\index{протокол!SMTP}, POP3\index{протокол!POP3}, IMAP\index{протокол!IMAP} и~т.\,д.) в SSL/TLS, к обозначению добавляют суффикс <<S>> (<<Secure>>): HTTPS\index{протокол!HTTPS}, FTPS\index{протокол!FTPS}, POP3S\index{протокол!POP3S}, IMAPS\index{протокол!IMAPS} и~т.\,д.

Протокол обеспечивает следующее:
\begin{itemize}
    \item Одностороннюю или взаимную аутентификацию клиента и сервера по открытым ключам сертификата X.509. В Интернете, как правило, делается \textbf{односторонняя} аутентификация веб-сервера браузеру клиента, то есть только веб-сервер предъявляет сертификат (открытый ключ и ЭП к нему от вышележащего УЦ).
    \item Создание сеансовых симметричных ключей для шифрования и кода аутентификации сообщения для передачи данных в обе стороны.
    \item Конфиденциальность\index{конфиденциальность} -- блочное или потоковое шифрование передаваемых данных в обе стороны.
    \item Целостность\index{целостность} -- аутентификацию отправляемых сообщений в обе стороны имитовставкой\index{имитовставка} $\HMAC(K,M)$, описанной ранее.
\end{itemize}

Рассмотрим протокол TLS последней версии 1.2.


\subsection{Протокол <<рукопожатия>>}

Протокол <<рукопожатия>> (\langen{Handshake Protocol}) производит аутентификацию и создание сеансовых ключей между клиентом $C$ и сервером $S$.

\begin{enumerate}
    \item $C \rightarrow S$:
        \begin{enumerate}
            \item ClientHello: ~ 1) URI сервера, ~ 2) одноразовая метка $N_C$\index{одноразовая метка}, ~ 3) поддерживаемые алгоритмы шифрования, кода аутентификации сообщений, хэширования, ЭП и сжатия.
        \end{enumerate}

    \item $C \leftarrow S$:
        \begin{enumerate}
            \item ServerHello: одноразовая метка $N_S$, поддерживаемые алгоритмы сервера.

            После обмена набором желательных алгоритмов сервер и клиент по единому правилу выбирают общий набор алгоритмов.
            \item Server Certificate: сертификат X.509v3 сервера с запрошенным URI (URI нужен в случае нескольких виртуальных веб-серверов с разными URI на одном узле c одним IP-адресом).
            \item Server Key Exchange Message: информация для создания предварительного общего секрета $premaster$ длиной 48 байтов в виде: ~ 1) обмена по протоколу Диффи~---~Хеллмана\index{протокол!Диффи~---~Хеллмана} с клиентом (сервер отсылает $(g, g^a)$), ~ 2) по другому алгоритму с открытым ключом, ~ 3) разрешения клиенту выбрать ключ.
            \item Электронная подпись к Server Key Exchange Message на ключе сертификата сервера для аутентификации сервера клиенту.
            \item Certificate Request: опциональный запрос сервером сертификата клиента.
            \item Server Hello Done: идентификатор конца транзакции.
        \end{enumerate}

    \item $C \rightarrow S$:
        \begin{enumerate}
            \item Client Certificate: сертификат X.509v3 клиента, если он был запрошен сервером.
            \item Client Key Exchange Message: информация для создания предварительного общего секрета $premaster$ длиной 48 байтов в виде: ~ 1) либо обмена по протоколу Диффи~---~Хеллмана\index{протокол!Диффи~---~Хеллмана} с сервером (клиент отсылает $g^b$, в результате обе стороны вычисляют ключ $premaster = g^{ab}$), ~ 2) либо по другому алгоритму, ~ 3) либо ключа, выбранного клиентом и зашифрованного на открытом ключе из сертификата сервера.
            \item Электронная подпись к Client Key Exchange Message на ключе сертификата клиента для аутентификации клиента серверу (если клиент использует сертификат).
            \item Certificate Verify: результат проверки сертификата сервера.
            \item Change Cipher Spec: уведомление о смене сеансовых ключей.
            \item Finished: идентификатор конца транзакции.
        \end{enumerate}

    \item $C \leftarrow S$:
        \begin{enumerate}
            \item Change Cipher Spec: уведомление о смене сеансовых ключей.
            \item Finished: идентификатор конца транзакции.
        \end{enumerate}
\end{enumerate}

%      http://tools.ietf.org/html/rfc5246#page-37

%      struct {
%          ProtocolVersion client_version;
%          Random random;
%          SessionID session_id;
%          CipherSuite cipher_suites<2..2^16-2>;
%          CompressionMethod compression_methods<1..2^8-1>;
%          select (extensions_present) {
%              case false:
%                  struct {};
%              case true:
%                  Extension extensions<0..2^16-1>;
%          };
%      } ClientHello;

%      struct {
%          ProtocolVersion server_version;
%          Random random;
%          SessionID session_id;
%          CipherSuite cipher_suite;
%          CompressionMethod compression_method;
%          select (extensions_present) {
%              case false:
%                  struct {};
%              case true:
%                  Extension extensions<0..2^16-1>;
%          };
%      } ServerHello;

%      struct {
%          ASN.1Cert certificate_list<0..2^24-1>;
%      } Certificate;

%      struct {
%          select (KeyExchangeAlgorithm) {
%              case dh_anon:
%                  ServerDHParams params;
%              case dhe_dss:
%              case dhe_rsa:
%                  ServerDHParams params;
%                  digitally-signed struct {
%                      opaque client_random[32];
%                      opaque server_random[32];
%                      ServerDHParams params;
%                  } signed_params;
%              case rsa:
%              case dh_dss:
%              case dh_rsa:
%                  struct {} ;
%                 /* message is omitted for rsa, dh_dss, and dh_rsa */
%              /* may be extended, e.g., for ECDH -- see [TLSECC] */
%          };
%      } ServerKeyExchange;

%      struct {
%          ClientCertificateType certificate_types<1..2^8-1>;
%          SignatureAndHashAlgorithm
%            supported_signature_algorithms<2^16-1>;
%          DistinguishedName certificate_authorities<0..2^16-1>;
%      } CertificateRequest;

%      struct {
%          select (KeyExchangeAlgorithm) {
%              case rsa:
%                  EncryptedPreMasterSecret;
%              case dhe_dss:
%              case dhe_rsa:
%              case dh_dss:
%              case dh_rsa:
%              case dh_anon:
%                  ClientDiffieHellmanPublic;
%          } exchange_keys;
%      } ClientKeyExchange;

%      struct {
%           digitally-signed struct {
%               opaque handshake_messages[handshake_messages_length];
%           }
%      } CertificateVerify;

%      struct {
%          opaque verify_data[verify_data_length];
%      } Finished;

Одноразовая метка $N_C$ состоит из 32 байтов. Первые 4 байта содержат текущее время (gmt\_unix\_time), оставшиеся байты -- псевдослучайную последовательность, которую формирует криптографически стойкий генератор псевдослучайных чисел.

Предварительный общий секрет $premaster$ длиной 48 байтов вместе с одноразовыми метками используется как инициализирующее значение генератора $PRF$ для получения общего секрета $master$ тоже длиной 48 байтов:
    \[ master = PRF(premaster, ~\text{текст ''master secret''}, ~ N_C + N_S) .\]

И, наконец, уже из секрета $master$ таким же способом генерируется 6 окончательных сеансовых ключей, следующих друг за другом в битовой строке:
    \[ \{ (K_{E,1} ~\|~ K_{E,2}) ~\|~ (K_{\MAC,1} ~\|~ K_{\MAC,2}) ~\|~ (IV_1 ~\|~ IV_2) \} = \]
        \[ = PRF(master, ~\text{текст ''key expansion''}, ~ N_C + N_S), \]
где $K_{E,1}, ~ K_{E,2}$ -- два ключа симметричного шифрования, ~ $K_{\MAC,1}, ~ K_{\MAC,2}$ -- два ключа имитовставки\index{имитовставка}, ~ $IV_1, ~IV_2$ -- два инициализирующих вектора режима сцепления блоков\index{вектор инициализации}. Ключи с индексом 1 используются для коммуникации от клиента к серверу, с индексом 2 -- от сервера к клиенту.


\subsection{Протокол записи}

Протокол записи (\langen{Record Protocol}) определяет формат TLS-пакетов для вложения в TCP-пакеты.

\begin{enumerate}
    \item Исходными сообщениями $M$ для шифрования являются пакеты протокола следующего уровня в модели OSI: HTTP\index{протокол!HTTP}, FTP\index{протокол!FTP}, IMAP\index{протокол!IMAP} и~т.\,д.
    \item Сообщение $M$ разбивается на блоки $m_i$ размером не более 16 кБ.
    \item Блоки $m_i$ сжимаются алгоритмом компрессии в блоки $z_i$.
    \item Вычисляется имитовставка\index{имитовставка} для каждого блока $z_i$ и добавляется в конец блоков: $a_i = z_i ~\|~ \HMAC(K_{\MAC}, z_i)$.
    \item Блоки $a_i$ шифруются симметричным алгоритмом с ключом $K_E$ в некотором режиме сцепления блоков с инициализирующим вектором $IV$ в полное сжатое аутентифицированное зашифрованное сообщение $C$.
    \item К шифротексту $C$ добавляется заголовок протокола записи TLS, и в результате получается TLS-пакет для вложения в TCP-пакет.
\end{enumerate}


\input{ipsec}

\section[Защита персональных данных в мобильной связи]{Защита персональных данных в \protect\\ мобильной связи}

\input{gsm2}

\input{gsm3}

%\section{Беспроводная сеть Wi-Fi}
%\subsection{WPA-PSK2, 802.11n, Radix?}
%\subsection{Wimax 802.16(?)}

\chapter{Аутентификация пользователя}


\section{Многофакторная аутентификация}

Для защищённых приложений применяется \textbf{многофакторная} аутентификация одновременно по факторам различной природы:
\begin{enumerate}
    \item Свойство, которым обладает субъект. Например: биометрия, природные уникальные отличия (лицо, радужная оболочка глаз, папиллярные узоры, последовательность ДНК).
    \item Знание -- информация, которую знает субъект. Например: пароль, PIN-код.
    \item Владение -- вещь, которой обладает субъект. Например: электронная или магнитная карта, флеш-память.
%    \item Факторы присвоения. Например, номер машины, RFID-метка.
\end{enumerate}

В обычных массовых приложениях из-за удобства использования применяется аутентификация только по \textbf{паролю}\index{пароль}, который является общим секретом пользователя и информационной системы. Биометрическая аутентификация по отпечаткам пальцев применяется существенно реже. Как правило, аутентификация по отпечаткам пальцев является дополнительным, а не вторым обязательным фактором (тоже из-за удобства её использования).

%Так же явно или неявно используется аутентификация по факторам:
%\begin{enumerate}
%    \item Социальная сеть. Доверие к индивидууму в личном или интернет общении, на основании общих связей.
%    \item Географическое положение. Например, для проверки оплаты товаров по кредитной карте.
%    \item Время. Доступ к сервисам или местам только в определённое время.
%    \item И др.
%\end{enumerate}


\section[Энтропия и криптостойкость паролей]{Энтропия и криптостойкость \protect\\ паролей}

Стандартный набор символов паролей, которые можно набрать на клавиатуре, используя английские буквы и небуквенные символы, состоит из $D=94$ символов. При длине пароля $L$ символов и предположении равновероятного использования символов энтропия паролей равна
    \[ H = L \log_2 D. \]

Клод Шеннон, исследуя энтропию символов английского текста, изучал вероятность успешного предсказания людьми следующего символа по первым нескольким символам слов или текста. В результате Шеннон получил оценку энтропии первого символа $s_1$ текста порядка $H(s_1) \approx 4{,}6$--$4{,}7$ бит/символ и оценки энтропий последующих символов, постепенно уменьшающиеся до $H(s_9) \approx 1{,}5$ бит/символ для 9-го символа. Энтропия для длинных текстов литературных произведений получила оценку $H(s_\infty) \approx 0{,}4$ бит/символ.

Статистические исследования баз паролей показывают, что наиболее часто используются буквы <<a>>, <<e>>, <<o>>, <<r>> и цифра <<1>>.

NIST использует следующие рекомендации для оценки энтропии паролей\index{энтропия!пароля}, создаваемых людьми.
\begin{enumerate}
    \item Энтропия первого символа $H(s_1) = 4$ бит/символ.
    \item Энтропия со 2-го по 8-й символы $H(s_{i}) = 2$ бит/символ, $2 \leq i \leq 8$.
    \item Энтропия с 9-го по 20-й символы $H(s_{i} = 1{,}5$ бит/символ, $9 \leq i \leq 20$.
    \item Энтропия с 21-го символа $H(s_{i}) = 1$ бит/символ, $i \geq 21$.
    \item Проверка композиции на использование символов разных регистров и небуквенных символов добавляет до 6 бит энтропии пароля.
    \item Словарная проверка на слова и часто используемые пароли добавляет до 6 бит энтропии для коротких паролей. Для 20-символьных и более длинных паролей прибавка к энтропии 0 бит.
\end{enumerate}

Для оценки энтропии пароля нужно сложить энтропии символов $H(s_i)$ и сделать дополнительные надбавки, если пароль удовлетворяет тестам на композицию и отсутствие в словаре.

\begin{table}[!ht]
    \centering
    \caption{Оценка NIST предполагаемой энтропии паролей\label{tab:password-entropy}}
    \resizebox{\textwidth}{!}{ \begin{tabular}{|c||c|c|c||c|}
        \hline
        \multirow{2}{*}{\parbox{1.5cm}{Длина пароля, символы}} & \multicolumn{3}{|c||}{\parbox{6cm}{Энтропия паролей пользователей по критериям NIST}} & \multirow{2}{*}{\parbox{2.5cm}{Энтропия случайных равновероятных паролей}} \\
        \cline{2-4}
        & \parbox{1.5cm}{Без проверок} & \parbox{2cm}{Словарная проверка} & \parbox{2.5cm}{Словарная и композиционная проверка} & \\
        \hline
        4  & 10 & 14 & 16 & 26.3 \\
        6  & 14 & 20 & 23 & 39.5 \\
        8  & 18 & 24 & 30 & 52.7 \\
        10 & 21 & 26 & 32 & 65.9 \\
        12 & 24 & 28 & 34 & 79.0 \\
        16 & 30 & 32 & 38 & 105.4 \\
        20 & 36 & 36 & 42 & 131.7 \\
        24 & 40 & 40 & 46 & 158.0 \\
        30 & 46 & 46 & 52 & 197.2 \\
        40 & 56 & 56 & 62 & 263.4 \\
        \hline
    \end{tabular} }
\end{table}

В таблице~\ref{tab:password-entropy} приведена оценка NIST на величину энтропии пользовательских паролей в зависимости от их длины, и приведено сравнение с энтропией случайных паролей с равномерным распределением символов из набора в $D=94$ символов клавиатуры. Вероятное число попыток для подбора пароля составляет $O(2^H)$. Из таблицы видно, что по критериям NIST энтропия реальных паролей в 2--4 раза меньше энтропии случайных паролей с равномерным распределением символов.

\example
Оценим общее количество существующих паролей. Население Земли -- 7 млрд. Предположим, что всё население использует компьютеры и Интернет, и у каждого человека по 10 паролей. Общее количество существующих паролей -- $7 \cdot 10^{10} \approx 2^{36}$.

Имея доступ к наиболее массовым интернет-сервисам с количеством пользователей десятки и сотни миллионов, в которых пароли часто хранятся в открытом виде из-за необходимости обновления ПО и, в частности, выполнения аутентификации, мы:
\begin{enumerate}
	\item имеем базу паролей, покрывающую существенную часть пользователей; 
	\item можем статистически построить правила генерирования паролей.
\end{enumerate}

Даже если пароль хранится в защищённом виде, то при вводе пароль, как правило, в открытом виде пересылается по Интернету, и все преобразования пароля для аутентификации осуществляет интернет-сервис, а не веб-браузер. Следовательно, интернет-сервис имеет доступ к исходному паролю.
\exampleend

В 2002 г. был подобран ключ для 64-битного блочного шифра RC5 сетью \texttt{distributed.net} персональных компьютеров, выполнявших вычисления в фоновом режиме. Суммарное время вычислений всех компьютеров -- 1757 дней, было проверено 83\% пространства всех ключей. Это означает, что пароли с оценочной энтропией менее 64 бит, то есть \emph{все пароли} до 40 символов по критериям NIST, могут быть подобраны в настоящее время. Конечно, с оговорками на то, что 1) нет ограничений на количество и скорость попыток аутентификаций, 2) алгоритм генерации вероятных паролей эффективен.

Строго говоря, использование даже 40-символьного пароля для аутентификации или в качестве ключа блочного шифрования является небезопасным.


\subsubsection{Число паролей}

Приведём различные оценки числа паролей, создаваемых людьми.

Пароли, создаваемые людьми, основаны на словах или закономерностях естественного языка. В английском языке всего около $1\ 000\ 000 \approx 2^{20}$ слов, включая термины.

%http://www.springerlink.com/content/bh216312577r6w64/fulltext.pdf
%http://www.antimoon.com/forum/2004/4797.htm

Используемые слоги английского языка имеют вид V, CV, VC, CVV, VCC, CVC, CCV, CVCC, CVCCC, CCVCC, CCCVCC, где C -- согласная (consonant), V -- гласная (vowel). 70\% слогов имеют структуру VC или CVC. Общее число слогов $S = 8000 - 12000$. Средняя длина слога -- 3 буквы.

Предполагая равновероятное распределение всех слогов английского языка, для числа паролей из $r$ слогов получим верхнюю оценку
    \[ N_1 = S^r = 2^{13 r} \approx 2^{4.3 L_1}. \]
Средняя длина паролей составит
    \[ L_1 \approx 3 r. \]

Теперь предположим, что пароли могут состоять только из 2--3 буквенных слогов вида CV, VC, CVV, VCC, CVC, CCV с равновероятным распределением символов. Подсчитаем число паролей $N_2$, которые могут быть построены из $r$ таких слогов. В английском алфавите число гласных букв $n_v = 10, согласных n_c = 16, n = n_v + n_c = 26$. Верхняя оценка числа $r$-слоговых паролей:
    \[ N_2 = (n_c n_v + n_v n_c + n_c n_v n_v + n_v n_c n_c + n_c n_v n_c + n_c n_c n_v)^r \approx \]
        \[ \approx \left( n_c n_v(3 n_c + n_v) \right)^r, \]
    \[ N_2 \approx \left( \frac{n^3}{2} \right)^r \approx 2^{13 r} \approx 2^{4.3 L_2}. \]
Средняя длина паролей:
    \[ L_2 = \frac{n_c n_v(2 + 2 + 3 n_v + 3 n_c + 3 n_c + 3 n_c)}{n_c n_v (1 + 1 + n_v + n_c + n_c + n_c)} \cdot r \approx 3 r. \]

Как видно, получились одинаковые оценки числа и длины паролей.

Подсчитаем верхние оценки числа паролей из $L$ символов, предполагая равномерное распределение символов из алфавита мощностью $D$ символов: a) $D_1 = 26$ строчных букв, б) все $D_2 = 94$ печатных символа клавиатуры (латиница и небуквенные символы):
    \[ N_3 = D_1^L \approx 2^{4.7 L}, \]
    \[ N_4 = D_2^L \approx 2^{6.6 L}. \]

\begin{table}[!ht]
    \centering
    \caption{Различные верхние оценки числа паролей\label{tab:password-number}}
    \resizebox{\textwidth}{!}{ \begin{tabular}{|c||c|c|c|}
        \hline
        \multirow{2}{*}{\parbox{1.5cm}{Длина пароля}} & \multicolumn{3}{|c|}{Число паролей} \\
        \cline{2-4}
            & \parbox{3cm}{На основе слоговой композиции} &
            \parbox{3cm}{Алфавит $D=26$ символов} &
            \parbox{3cm}{Алфавит $D=94$ символа} \\
        \hline \hline
        6  & $2^{26}$ & $2^{28}$ & $2^{39}$ \\
        9  & $2^{39}$ & $2^{42}$ & $2^{59}$ \\
        12 & $2^{52}$ & $2^{56}$ & $2^{79}$ \\
        15 & $2^{65}$ & $2^{71}$ & $2^{98}$ \\
        \hline
        21 & $2^{91}$ & $2^{99}$ & $2^{137}$ \\
        \hline
        39 & $2^{169}$ & $2^{183}$ & $2^{256}$ \\
        \hline
    \end{tabular} }
\end{table}

Из таблицы~\ref{tab:password-number} видно, что при доступном объёме вычислений в $2^{60 \ldots 70}$ операций, пароли вплоть до 15 символов, построенные на словах, слогах, изменениях слов, вставках цифр, небольшом изменении регистров и других простейших модификациях, могут быть найдены перебором на кластере (или ПК) в настоящее время.

Для достижения криптостойкости паролей, сравнимой со 128- или 256-битовым секретным ключом, требуется выбирать пароль из 20 и 40 символов соответственно, что, как правило, не реализуется из-за сложности запоминания и ввода без ошибок.


%Подсчитаем число паролей $N_1$, которые могут могут построены из $r$ ~ 2-3 буквенных слогов: $cv, vc, ccv, cvc, vcc$, где $c$ -- согласная, $v$ -- гласная. В английском алфавите $n_v = 10, n_c = 16, n = n_v + n_c = 26$. Число паролей
%    \[ N_1 = \left( n_v n_c (1 + 1 + n_c + n_c + n_c) \right)^r \approx 3^r n_v^r n_c^{2r}. \]
%Средняя длина паролей
%    \[ L = r \left( \frac{2 + 2 + 3 n_c + 3 n_c + 3 n_c}{1 + 1 + n_c + n_c + n_c} \right) \approx 3r. \]
%
%%Учтем, что $b \leq r$ символов могут быть заглавными: $N_1 \rightarrow N_2 < N_1 \binom{L}{b} \left( \frac{n}{n_v} \right)^b$. Вставим $d$ цифр в случайные места: $N_2 \rightarrow N_3 = N_2 (10 (1 + L))^d \approx N_2 (10 L)^d$.
%%
%%Общее число паролей
%%    \[ N = N_3 = 3^r 10^r 16^{2r} \binom{3r}{b} 2.6^b \left(10 \cdot 3 r \right)^d. \]
%%
%%\begin{table}[!ht]
%%    \centering
%%    \small
%%    \begin{tabular}{|c|c|c|c|c||cr|}
%%        \hline
%%        \parbox{1.3cm}{Слогов, $r$} & \parbox{1.8cm}{Заглавных букв, $b$} & \parbox{1.5cm}{Вставок цифр, $d$} & \parbox{2.8cm}{Средняя длина пароля, $L+d$} & \parbox{3cm}{Верхняя оценка числа паролей $N$} & \multicolumn{2}{|c|}{\parbox{3.2cm}{Число всех паролей}} \\
%%        \hline
%%        $2$ & $0$ & $0$ & $6$ & $2^{26}$ & $2^{36}$ & a-z \\
%%        $2$ & $2$ & $0$ & $6$ & $2^{32}$ & $2^{48}$ & A-Z, a-z \\
%%        $2$ & $2$ & $2$ & $8$ & $2^{45}$ & $2^{48}$ & A-Z, a-z, 0-9 \\
%%        \hline
%%        $3$ & $0$ & $0$ & $9$ & $2^{39}$ & $2^{54}$ & a-z \\
%%        $3$ & $3$ & $0$ & $9$ & $2^{49}$ & $2^{54}$ & A-Z, a-z \\
%%        $3$ & $3$ & $2$ & $11$ & $2^{63}$ & $2^{65}$ & A-Z, a-z, 0-9 \\
%%        \hline
%%        $4$ & $0$ & $0$ & $12$ & $2^{52}$ & $2^{93}$ & a-z \\
%%        $4$ & $3$ & $0$ & $12$ & $2^{64}$ & $2^{186}$ & A-Z, a-z \\
%%        $4$ & $3$ & $2$ & $14$ & $2^{78}$ & $2^{222}$ & A-Z, a-z, 0-9 \\
%%        \hline
%%    \end{tabular}
%%    \caption{Сравнение верхней оценки числа паролей, построенных на слогах, со всем доступным множеством паролей.}
%%    \label{tab:password-number}
%%\end{table}
%
%Учтем, что $b$ символов в пароле могут быть взяты не из 26-символьного алфавита строчных букв, а из всего алфавита в $D=94$ печатных символа клавиатуры (латиница и небуквенные символы):
%\[
%    \begin{array}{ll}
%    b=1 & N_1 \rightarrow N_2 = \frac{n_v}{n_v+n_c} 3^r n_v^{r-1} n_c^{2r} \cdot L. \]
%
%    \[ N_1 \rightarrow N_2 < N_1 \binom{L}{b} \left( \frac{D}{n_v} \right)^b. \]
%
%
%
%Общее число паролей
%    \[ N < 3^r n_v^r n_c^{2r} \binom{L}{b} \left( \frac{D}{n_v} \right)^b = 3^r 10^r 16^{2r} \binom{3r}{b} \left( \frac{94}{10} \right)^b. \]
%
%\begin{table}[!ht]
%    \centering
%    \small
%    \begin{tabular}{|c|c|c|c||cr|}
%        \hline
%        \parbox{1.5cm}{Слогов, $r$} & \parbox{3cm}{Средняя длина пароля, $L$} & \parbox{3cm}{Символов из всего алфавита, $b$} & \parbox{3cm}{Верхняя оценка числа паролей $N$} & \multicolumn{2}{|c|}{\parbox{3.2cm}{Число всех паролей, $D^L$}} \\
%        \hline
%        \multirow{3}{*}{2} & \multirow{3}{*}{6} & $0$ & $2^{26}$ & $2^{28}$ & a-z \\
%        & & $1$ & $2^{32}$ & $2^{34}$ & A-Z, a-z \\
%        & & $3$ & $2^{40}$ & $2^{39}$ & Весь алфавит \\
%        \hline
%        \multirow{3}{*}{3} & \multirow{3}{*}{9} & $0$ & $2^{39}$ & $2^{42}$ & a-z \\
%        & & $2$ & $2^{50}$ & $2^{51}$ & A-Z, a-z \\
%        & & $4$ & $2^{59}$ & $2^{59}$ & Весь алфавит \\
%        \hline
%        \multirow{3}{*}{4} & \multirow{3}{*}{12} & $0$ & $2^{52}$ & $2^{56}$ & a-z \\
%        & & $3$ & $2^{69}$ & $2^{68}$ & A-Z, a-z \\
%        & & $6$ & $2^{81}$ & $2^{77}$ & Весь алфавит \\
%        \hline
%    \end{tabular}
%    \caption{Сравнение верхней оценки числа паролей, построенных на слогах, со всем доступным множеством паролей в алфавите из $D$ символов.}
%    \label{tab:password-number}
%\end{table}
%
%Из таблицы~\ref{tab:password-number} видно, что при доступном объёме вычислений в $2^{60 \ldots 70}$ операций, пароли вплоть до 12 символов, построенные на словах, слогах, изменениях слов, вставках цифр, небольшого изменения регистров и другой простейшей обфускации, могут быть найдены перебором на кластере (или ПК) в настоящее время.


\subsubsection{Атака для подбора паролей и ключей шифрования}

В схемах аутентификации по паролю иногда используется хэширование и хранение хэша пароля на сервере. В таких случаях применима словарная атака или атака с применением заранее вычисленных таблиц для ускорения поиска.

Для нахождения пароля, прообраза хэш-функции, или для нахождения ключа блочного шифрования по атаке с выбранным шифротекстом (для одного и того же известного открытого текста и соответствующего шифротекста) может быть применён метод перебора с балансом между памятью и временем вычислений. Самый быстрый метод радужных таблиц\index{радужные таблицы} (\langen{rainbow tables}, 2003~г., \cite{Oechslin:2003}) заранее вычисляет следующие цепочки и хранит результат в памяти.

Для нахождения пароля, прообраза хэш-функции $H$, цепочка строится как
    \[ M_0 \xrightarrow{H(M_0)} h_0 \xrightarrow{R_0(h_0)} M_1 \ldots M_t \xrightarrow{H(M_t)} h_t \xrightarrow{R_t(h_t)} M_{t+1}, \]
где $R_i(h)$ -- функция редуцирования, преобразования хэша в пароль для следующего хэширования.

Для нахождения ключа блочного шифрования для одного и того же известного открытого текста $M$ таблица строится как
    \[ K_0 \xrightarrow{E_{K_0}(M)} c_0 \xrightarrow{R_0(c_0)} K_1 \ldots K_t \xrightarrow{E_{K_t}(M)} c_t \xrightarrow{R_t(c_t)} K_{t+1}, \]
где $R_i(c)$ -- функция редуцирования, преобразования шифротекста в новый ключ.

Функция редуцирования $R_i$ зависит от номера итерации, чтобы избежать дублирующихся подцепочек, которые возникают в случае коллизий между значениями в разных цепочках в разных итерациях, если $R$ постоянна. Rainbow-таблица размера $(m \times 2)$ состоит из строк $(M_{0,j}, M_{t+1,j})$ или $(K_{0,j}, K_{t+1,j})$, вычисленных для разных значений стартовых паролей $M_{0,j}$ или $K_{0,j}$ соответственно.

Опишем атаку на примере нахождения прообраза $\overline{M}$ хэша $\overline{h} = H(\overline{M})$. На первой итерации исходный хэш $\overline{h}$ редуцируется в сообщение $\overline{h} \xrightarrow{R_t(\overline{h})} \overline{M}_{t+1} $ и сравнивается со всеми значениями последнего столбца $M_{t+1,j}$ таблицы. Если нет совпадения, переходим ко второй итерации. Хэш $\overline{h}$ дважды редуцируется в сообщение $\overline{h} \xrightarrow{R_{t-1}(\overline{h})} \overline{M}_t \xrightarrow{H(\overline{M}_t)} \overline{h}_t \xrightarrow{R_t(\overline{h}_t)} \overline{M}_{t+1}$ и сравнивается со всеми значениями последнего столбца $M_{t+1,j}$ таблицы. Если не совпало, то переходим к третьей итерации и~т.\,д. Если для $r$-кратного редуцирования сообщение $\overline{M}_{t+1}$ содержится в таблице во втором столбце, то из совпавшей строки берётся $M_{0,j}$, и вся цепочка пробегается заново для поиска искомого сообщения $\overline{M}: ~ \overline{h} = H(\overline{M})$.

Найдём вероятность нахождения пароля в таблице. Пусть мощность множества всех паролей $N$. Изначально в столбце $M_{0,j}$ содержится $m_0 = m$ различных паролей. Предполагая наличие случайного отображения с пересечениями паролей $M_{0,j} \rightarrow M_{1,j}$, в $M_{1,j}$ будет $m_1$ различных паролей. Согласно задаче о размещении,
\[
    m_{i+1} = N \left( 1 - \left( 1 - \frac{1}{N} \right)^{m_i} \right) \approx N \left( 1 - e^{-\frac{m_i}{N}} \right).
\]
Вероятность нахождения пароля
\[
    P = 1 - \prod \limits_{i=1}^t \left( 1 - \frac{m_i}{N} \right).
\]

Чем больше таблица из $m$ строк, тем больше шансов найти пароль или ключ, выполнив в наихудшем случае   $O \left( m \frac{t(t+1)}{2} \right)$ операций.

Примеры применения атаки на хэш-функциях $\textrm{MD5}$\index{хэш-функция!MD5}, $\textrm{LM} \sim \textrm{DES}_{\textrm{Password}} (\textrm{const})$ приведены в таблице~\ref{tab:rainbow-tables}.

\begin{table}[!ht]
    \centering
    \caption{Атаки на радужных таблицах на \emph{одном} ПК\label{tab:rainbow-tables}}
    \resizebox{\textwidth}{!}{ \begin{tabular}{|c|c|c|c|c|c|c|}
        \hline
        \multirow{2}{*}{\parbox{1.0cm}{Длина, биты}} & \multicolumn{3}{|c|}{Пароль или ключ} &
            \multicolumn{3}{|c|}{Радужная таблица} \\
        \cline{2-7}
        & \parbox{1.2cm}{Длина, симв.} & \parbox{1cm}{Множе- ство} & \parbox{1cm}{Мощ- ность} &
            объём & \parbox{1.5cm}{Время вычисления таблиц} & \parbox{1.3cm}{Время поиска} \\
        \hline \hline
        \multicolumn{7}{|c|}{Хэш LM} \\
        \hline
        \multirow{3}{*}{$2 \times 56$} & \multirow{3}{*}{14} &
            A--Z & $2^{33}$ & 610 MB &  & 6 с \\
        & & A--Z, 0-9 & $2^{36}$ & 3 GB &  & 15 с \\
        & & все & $2^{43}$ & 64 GB & \parbox{1.5cm}{несколько лет} & 7 мин \\
        \hline \hline
        \multicolumn{7}{|c|}{Хэш MD5} \\
        \hline
        128 & 8 & A-Z, 0-9 & $2^{41}$ & 36 GiB & - & 4 мин \\
        \hline
    \end{tabular} }
\end{table}


\section{Аутентификация по паролю}

Из-за малой энтропии пользовательских паролей во всех системах регистрации и аутентификации пользователей применяется специальная политика безопасности. Типичные минимальные требования:
\begin{enumerate}
    \item Длина пароля от 8 символов. Использование разных регистров и небуквенных символов в паролях. Запрет паролей из словаря или часто используемых паролей. Запрет паролей в виде дат, номеров машин и других номеров.
    \item Ограниченное время действия пароля. Обязательная смена пароля по истечении срока действия.
    \item Блокирование возможности аутентификации после нескольких неудачных попыток. Ограниченное число актов аутентификаций в единицу времени. Временная задержка перед выдачей результата аутентификации.
\end{enumerate}

Дополнительные рекомендации (требования) пользователям:
\begin{enumerate}
    \item Не использовать одинаковые или похожие пароли для разных систем, таких как электронная почта, вход в ОС, электронная платёжная система, форумы, социальные сети. Пароль часто передаётся в открытом виде по сети. Пароль доступен администратору системы, возможны утечки конфиденциальной информации с серверов. Поэтому следует стараться выбирать случайные стойкие пароли.
    \item Не записывать пароли. Никому не сообщать пароль, даже администратору. Не передавать пароли по почте, телефону, Интернету и~т.\,д.
    \item Не использовать одну и ту же учётную запись для разных пользователей, даже в виде исключения.
    \item Всегда блокировать компьютер, когда пользователь отлучается от него, даже на короткое время.
\end{enumerate}

\input{os_passwords}

\input{http_auth}

\chapter{Программные уязвимости}

\input{security_models}

\input{os_access_controls}

\section{Виды программных уязвимостей}

\textbf{Вирусом} называется самовоспроизводящаяся часть кода (подпрограмма)\index{вирус}, которая встраивается в носители (другие программы) для своего исполнения и распространения. Вирус не может исполняться и передаваться без своего носителя.

\textbf{Червём} называется самовоспроизводящаяся отдельная (под)программа\index{червь}, которая может исполняться и распространяться самостоятельно, не используя программу-носитель.

Первой вехой в изучении компьютерных вирусов можно назвать 1949 год, когда Джон фон Нейман прочёл курс лекций в Университете Иллинойса под названием <<Теория самовоспроизводящихся машин>> (изданы в 1966~\cite{Neumann:1966}, переведены на русский язык издательством <<Мир>> в 1971 году~\cite{Neumann:1971}), в котором ввёл понятие самовоспроизводящихся механических машин. Первым сетевым вирусом считается вирус Creeper 1971 г., распространявшийся в сети ARPANET, предшественнике Интернета. Для его уничтожения был создан первый антивирус Reaper, который находил и уничтожал Creeper.

Первый червь для Интернета, червь Морриса 1988 г., уже использовал \emph{смешанные} атаки\index{атака!смешанная} для заражения UNIX машин~\cite{EichinRochlis:1988, Spafford:1989}. Сначала программа получала доступ к удалённому запуску команд, эксплуатируя уязвимости в сервисах \texttt{sendmail}, \texttt{finger} (с использованием атаки на переполнение буфера) или \texttt{rsh}. Далее с помощью механизма подбора паролей, червь получал доступ к локальным аккаунтам пользователей:
\begin{itemize}
    \item получение доступа к учётным записям с очевидными паролями:
		\begin{itemize}
			\item без пароля вообще;
			\item имя аккаунта в качестве пароля;
			\item имя аккаунта в качестве пароля, повторённое дважды;
			\item использование <<ника>> (\langen{nickname});
			\item фамилия (\langen{last name, family name});
			\item фамилия, записанная задом наперёд;
		\end{itemize}
		\item перебор паролей на основе встроенного словаря из 432 слов;
		\item перебор паролей на основе системного словаря \texttt{/usr/dict/words}.
\end{itemize}

\textbf{Программной уязвимостью}\index{программная уязвимость} называется свойство программы, позволяющее нарушить её работу. Программные уязвимости могут приводить к отказу в обслуживании (Denial of Service, DoS-атака)\index{атака!отказ в обслуживании}, утечке и изменению данных, появлению и распространению вирусов и червей.

Одной из распространённых атак для заражения персональных компьютеров является переполнение буфера в стеке. В интернет-сервисах наиболее распространённой программной уязвимостью в настоящее время является межсайтовый скриптинг (Cross-Site Scripting, XSS-атака)\index{атака!XSS}.

Наиболее распространённые программные уязвимости можно разделить на классы:
\begin{enumerate}
    \item Переполнение буфера -- копирование в буфер данных большего размера, чем длина выделенного буфера. Буфером может быть контейнер текстовой строки, массив, динамически выделяемая память и~т.\,д. Переполнение становится возможным, вследствие либо отсутствия контроля над длиной копируемых данных, либо из-за ошибок в коде. Типичная ошибка -- разница в 1 байт между размерами буфера и данных при сравнении.
    \item Некорректная обработка (парсинг) данных, введённых пользователем, является причиной большинства программных уязвимостей в веб-приложениях. Под обработкой понимаются:
        \begin{enumerate}
            \item проверка на допустимые значения и тип (числовые поля не должны содержать строки и~т.\,д.);
            \item фильтрация и экранирование специальных символов, имеющих значения в скриптовых языках или применяющихся для декодирования из одной текстовой кодировки в другую. Примеры символов: \texttt{\textbackslash}, \texttt{\%}, \texttt{<}, \texttt{>}, \texttt{"}, \texttt{'};
            \item фильтрация ключевых слов языков разметки и скриптов. Примеры: \texttt{script}, \texttt{JavaScript};
            \item декодирование различными кодировками при парсинге. Распространённый способ обхода системы контроля парсинга данных состоит в однократном или множественном последовательном кодировании текстовых данных в шестнадцатеричные кодировки \texttt{\%NN} ASCII и UTF-8. Например, браузер или веб-приложения производят $n$-кратные последовательные декодирования, в то время как система контроля делает $k$-кратное декодирование, $0 \leq k < n$, и, следовательно, пропускает закодированные запрещённые символы и слова.
        \end{enumerate}
    \item Некорректное использование синтаксиса функций. Например, \texttt{printf(s)} может привести к уязвимости записи в указанный адрес памяти. Если злоумышленник вместо обычной текстовой строки введёт в качестве \texttt{s = "текст некоторой длины\%n"}, то функция \texttt{printf()}, ожидающая первым аргументом строку формата \texttt{printf(fmt, \dots)}, обнаружив \texttt{\%n}, возьмёт значения из ячеек памяти, следующих перед текстовой строкой (устройство стека функции описано далее), и запишет в адрес памяти, равный считанному значению, количество выведенных символов на печать функцией \texttt{printf()}.
\end{enumerate}


\input{stack_overflow}

\input{xss}

\input{sql-injections}

%\chapter{Послесловие}
%Это должно быть что-то в виде заключения, объяснения, почему именно эти темы выбраны, насколько актуален материал с теоретической и практической точки зрения.


\appendix
\renewcommand{\thechapter}{\Asbuk{chapter}} % использование русских букв для нумерации приложений

\chapter{Математическое приложение}\label{chap:discrete-math}

\section{Общие определения}

Выражением $\mod n$ обозначается вычисление остатка от деления произвольного целого числа на целое число $n$. В полиномиальной арифметике эта операция означает вычисление остатка от деления многочленов.
%далее будем обозначать целые числа или операции с целыми числами, взятыми \textbf{по модулю}\index{модуль} целого числа $n$ (остаток от целочисленного деления). Примеры выражений:
    \[ a\mod n, \]
    \[ (a + b) c\mod n. \]
Равенство
    \[ a = b \mod n \]
означает, что выражения $a$ и $b$ равны (говорят также <<сравнимы>>, <<эквивалентны>>) по модулю $n$.

Множество
    \[ \{ 0, 1, 2, 3, \dots, n-1 \mod n\} \]
состоит из $n$ элементов, где каждый элемент $i$ представляет все целые числа, сравнимые с $i$ по модулю $n$.

Наибольший общий делитель (НОД) двух чисел $a,b$ обозначается $\gcd(a,b)$ (greatest common divisor).

Два числа $a,b$ называются взаимно простыми, если они не имеют общих делителей, кроме 1, то есть $\gcd(a,b) = 1$.

Выражение $a \mid b$ означает, что $a$ делит $b$.

\input{birthdays_paradox}

\section{Группы}\label{section-groups}
\selectlanguage{russian}

\subsection{Свойства групп}

\textbf{Группой}\index{группа} называется множество $\Gr$, на котором задана бинарная операция <<$\cdot$>>, удовлетворяющая следующим аксиомам:
\begin{enumerate}
    \item замкнутость:
        \[ \forall a,b \in \Gr: a \cdot b = c \in \Gr; \]
    \item ассоциативность:
        \[ \forall a,b,c \in \Gr: (a \cdot b) \cdot c = a \cdot (b \cdot c); \]
    \item существование единичного элемента:
        \[ \exists ~ e \in \Gr: e\cdot a = a \cdot e = a; \]
    \item существование обратного элемента:
        \[ \forall a \in \Gr ~ \exists ~ b \in \Gr: a \cdot b = b \cdot a = e. \]
\end{enumerate}
Если
    \[ \forall a,b \in \Gr: a \cdot b = b \cdot a, \]
то группа коммутативная.

Если операция в группе задана как умножение <<$\cdot$>>, то группа называется \textbf{мультипликативной}, $e = 1$, обратный элемент -- $a^{-1}$, возведение в степень $k$ -- $a^k$.

Если операция задана как сложение <<$+$>>, то группа называется \textbf{аддитивной}, $e = 0$, обратный элемент $-a$, сложение $k$ раз -- $ka$.

Подмножество группы, удовлетворяющее аксиомам группы, называется \textbf{подгруппой}\index{подгруппа}.

\textbf{Порядком} $|\Gr|$ \textbf{группы}\index{порядок группы} $\Gr$ называется число элементов в группе. Пусть группа мультипликативная. Для любого элемента $a \in \Gr$ выполняется $a^{|\Gr|} = 1$.

\textbf{Порядком элемента} $a$ называется минимальное натуральное число
    \[ \ord(a): a^{\ord(a)} = 1. \]
 Порядок элемента делит порядок группы:
    \[ \ord(a) \mid \left|\Gr\right|. \]


\subsection{Циклические группы}

\textbf{Генератором} $g \in \Gr$ называется элемент, \emph{порождающий} всю группу\index{генератор группы}:
    \[ \Gr = \{g, g^2, g^3, \ldots, g^{|\Gr|} = 1\}. \]

Группа, в которой существует генератор, называется \textbf{циклической}\index{группа!циклическая}.

Если конечная группа не циклическая, то в ней существуют циклические подгруппы, порождённые всеми элементами. Любой элемент $a$ группы порождает либо циклическую \emph{подгруппу}
    \[ \{ a, a^2, a^3, \dots, a^{\ord(a)} = 1 \} \]
порядка $\ord(a)$, если порядок элемента $\ord(a) < |\Gr|$, либо \emph{всю} группу
    \[ \Gr = \{ a, a^2, a^3, \dots, a^{|\Gr|} = 1 \}, \]
если порядок элемента равен порядку группы $\ord(a) = |\Gr|$. Порядок любой подгруппы, как и порядок элемента, делит порядок всей группы.

Представим циклическую группу через генератор $g$ как
    \[ \Gr = \{g, g^2, \ldots, g^{|\Gr|} = 1\} \]
и каждый элемент $g^i$ возведём в степени $1, 2, \ldots, |\Gr|$. Тогда
\begin{itemize}
    \item элементы $g^i$, для которых число $i$ взаимно просто с $|\Gr|$, породят снова всю группу
            \[ \Gr = \{ g^i, g^{2i}, g^{3i}, \dots, g^{|\Gr| i} = 1 \}, \]
        так как степени $\{i, 2i, 3i, \dots, |\Gr| i \}$ по модулю $|\Gr|$ образуют перестановку чисел $\{1, 2, 3, \dots, |\Gr|\}$; следовательно $g^i$ -- тоже генератор, число таких чисел $i$ по определению функции Эйлера $\varphi(|\Gr|)$ ($\varphi(n)$ -- количество взаимно простых с $n$ целых чисел в диапазоне $[1,n-1]$);
    \item элементы $g^i$, для которых $i$ имеют общие делители
            \[ d_i = \gcd(i, |\Gr|) \neq 1 \]
        c $|\Gr|$, породят подгруппы
            \[ \{ g^i, g^{2i}, g^{3i}, \dots, g^{\frac{i}{d_i} |\Gr|} = 1\}, \]
        так как степень последнего элемента кратна $|\Gr|$; следовательно такие $g^i$ образуют циклические подгруппы порядка $d_i$.
\end{itemize}

Из предыдущего утверждения следует, что число генераторов в циклической группе равно
    \[ \varphi(|\Gr|). \]

Для проверки, является ли элемент генератором всей группы, требуется знать разложение порядка группы $|\Gr|$ на множители. Далее элемент возводится в степени, равные всем делителям порядка группы, и сравнивается с единичным элементом $e$. Если ни одна из степеней не равна $e$, то этот элемент является примитивным элементом или генератором группы. В противном случае элемент будет генератором какой-либо подгруппы.

Задача разложения числа на множители является трудной для вычисления. На сложности её решения, например, основана криптосистема RSA\index{криптосистема!RSA}. Поэтому при создании больших групп желательно заранее знать разложение порядка группы на множители для возможности выбора генератора.


\subsection{Группа $\Z_p^*$}\label{section-group-multiplicative}

\textbf{Группой $\Z_p^*$} называется группа\index{группа!$\Z_p^*$}
    \[ \Z_p^* = \{1, 2, \dots, p-1 \mod p\}, \]
где $p$ -- простое\index{число!простое} число, операция в группе -- умножение $\ast$ по $\mod p$.

Группа $\Z_p^*$ -- \textbf{циклическая}, порядок
    \[ |\Z_p^*| = \varphi(p) = p - 1. \]
Число генераторов в группе --
    \[ \varphi(|\Z_p^*|) = \varphi(p-1). \]

Из того, что $\Z_p^*$ -- группа, для простого\index{число!простое} $p$ и любого $a \in [2, p-1] \mod p$ следует \textbf{малая теорема Ферма}\index{теорема!Ферма малая}:
    \[ a^{p-1} = 1 \mod p. \]
На малой теореме Ферма основаны многие тесты проверки числа на простоту.

\example 1
Рассмотрим группу $\Z_{19}^*$. Порядок группы -- 18. Делители: 2, 3, 6, 9. Является ли 12 генератором?
\[ \begin{array}{l}
    12^2 = -8 \mod 19, \\
    12^3 = -1 \mod 19, \\
    12^6 = 1 \mod 19, \\
\end{array} \]
12 -- генератор подгруппы 6 порядка. Является ли 13 генератором?
\[ \begin{array}{l}
    13^2 = -2 \mod 19, \\
    13^3 = -7 \mod 19, \\
    13^6 = -8 \mod 19, \\
    13^9 = -1 \mod 19, \\
    13^{18} = 1 \mod 19, \\
\end{array} \]
13 -- генератор всей группы.
\exampleend

\example 2
В таблице~\ref{tab:Zp-sample} приведён пример группы $\Z_{13}^*$. Число генераторов -- $\varphi(12) = 4$. Подгруппы --
    \[ \Gr^{(1)}, \Gr^{(2)}, \Gr^{(3)}, \Gr^{(4)}, \Gr^{(6)}, \]
верхний индекс обозначает порядок подгруппы.

\begin{table}[!ht]
    \centering
    \caption {Генераторы и циклические подгруппы группы $\Gr=\Z_{13}^*$\label{tab:Zp-sample}}
    \resizebox{\textwidth}{!}{ \begin{tabular}{|c|p{0.66\textwidth}|c|}
        \hline
        Элемент & Порождаемая группа или подгруппа & Порядок \\
        \hline
         1 & $\Gr^{(1)} = \{  1 \}$ & 1 \\
         2 & $\Gr = \{ 2, 4, 8, 3, 6, 12, 11, 9, 5, 10, 7, 1 \}$ & 12, ген. \\
         3 & $\Gr^{(3)} = \{ 3, 9, 1 \}$ & 3 \\
         4 & $\Gr^{(6)} = \{ 4, 3, 12, 9, 10, 1 \}$ & 6 \\
         5 & $\Gr^{(4)} = \{ 5, 12, 8, 1 \}$ & 4 \\
         6 & $\Gr = \{ 6, 10, 8, 9, 2, 12, 7, 3, 5, 4, 11, 1 \}$ & 12, ген. \\
         7 & $\Gr = \{ 7, 10, 5, 9, 11, 12, 6, 3, 8, 4, 2, 1 \}$ & 12, ген. \\
         8 & $\Gr^{(4)} = \{ 8, 12, 5, 1 \}$ & 4 \\
         9 & $\Gr^{(3)} = \{ 9, 3, 1 \}$ & 3 \\
        10 & $\Gr^{(6)} = \{ 10, 9, 12, 3, 4, 1 \}$ & 6 \\
        11 & $\Gr = \{ 11, 4, 5, 3, 7, 12, 2, 9, 8, 10, 6, 1 \}$ & 12, ген. \\
        12 & $\Gr^{(2)} = \{ 12, 1 \}$ & 2 \\
        \hline
    \end{tabular} }
\end{table}
\exampleend


\subsection{Группа $\Z_n^*$}

\textbf{Функция Эйлера}\index{функция!Эйлера} $\varphi(n)$ определяется как количество чисел, взаимно простых с $n$ в интервале от 1 до $n-1$.

Если $n=p$ -- простое\index{число!простое} число, то
    \[ \varphi(p) = p - 1, \]
    \[ \varphi(p^k) = p^k - p^{k-1} = p^{k-1}(p - 1). \]
Если $n$ -- составное число и
    \[ n = \prod \limits_{i} p_i^{k_i} \]
разложено на простые множители $p_i$, то
    \[ \varphi(n) = \prod \limits_{i} \varphi(p_i^{k_i}) =  \prod \limits_{i} p_i^{k_i - 1}(p_i - 1). \]

\textbf{Группой $\Z_n^*$} называется группа\index{группа!$\Z_n^*$}
    \[ \Z_n^* = \left\{ \forall a \in \left\{ 1, 2, \dots, n-1 \mod n \right\} : \gcd(a,n) = 1 \right\} \]
с операцией умножения $\ast$ по $\mod n$.

Порядок группы --
    \[ |\Z_n^*| = \varphi(n). \]
Группа $\Z_p^*$ -- частный случай группы $\Z_n^*$.

Если $n$ \emph{составное}\index{число!составное} (не простое) число, то группа $\Z_n^*$ \textbf{нециклическая}.

Из того, что $\Z_n^*$ -- группа, для любых $a \neq 0,\, n > 1:\, \gcd(a,n) = 1$ следует \textbf{теорема Эйлера}\index{теорема!Эйлера}:
    \[ a^{\varphi(n)} = 1 \mod n. \]

При возведении в степень, если $\gcd(a,n) = 1$, выполняется
    \[ a^b = a^{b \mod \varphi(n)} \mod n. \]

\example
В таблице~\ref{tab:Zn-sample} приведена нециклическая группа $\Z_{21}^*$ и её циклические подгруппы
    \[ \Gr^{(1)}, \Gr_1^{(2)}, \Gr_2^{(2)}, \Gr_3^{(2)}, \Gr_1^{(3)}, \Gr_1^{(6)}, \Gr_2^{(6)}, \Gr_3^{(6)}, \]
верхний индекс обозначает порядок подгруппы, нижний индекс нумерует различные подгруппы одного порядка.

\begin{table}[!ht]
    \centering
    \caption{Циклические подгруппы нециклической группы $\Z_{21}^*$\label{tab:Zn-sample}}
    \begin{tabular}{|c|l|c|}
        \hline
        Элемент & Порождаемая циклическая подгруппа & Порядок \\
        \hline
        1  & $\Gr^{(1)} = \{ 1 \}$ & 1 \\
        2  & $\Gr_1^{(6)} = \{ 2, 4, 8, 16, 11, 1 \}$ & 6 \\
        4  & $\Gr_1^{(3)} = \{ 4, 16, 1 \}$ & 3 \\
        5  & $\Gr_2^{(6)} = \{ 5, 4, 20, 16, 17, 1 \}$ & 6 \\
        8  & $\Gr_1^{(2)} = \{ 8, 1 \}$ & 2 \\
        10 & $\Gr_3^{(6)} = \{ 10, 16, 13, 4, 19, 1 \}$ & 6 \\
        11 & $\Gr_1^{(6)} = \{ 11, 16, 8, 4, 2, 1 \}$ & 6 \\
        13 & $\Gr_2^{(2)} = \{ 13, 1 \}$ & 2 \\
        16 & $\Gr_1^{(3)} = \{ 16, 4, 1 \}$ & 3 \\
        17 & $\Gr_2^{(6)} = \{ 17, 16, 20, 4, 5, 1 \}$ & 6 \\
        19 & $\Gr_3^{(6)} = \{ 19, 4, 13, 16, 10, 1 \}$ & 6 \\
        20 & $\Gr_3^{(2)} = \{ 20, 1 \}$ & 2 \\
        \hline
    \end{tabular}
\end{table}
\exampleend

\subsection{Конечные поля}\label{section-fields}

\textbf{Полем} называется множество $\F$, для которого\index{поле}:
\begin{itemize}
    \item заданы две бинарные операции, условно называемые операциями умножения <<$\cdot$>> и сложения <<$+$>>;
    \item выполняются аксиомы группы для операции <<сложения>>: \\
        1. замкнутость:
		\[\forall a, b \in \F: a + b \in \F;\]
        2. ассоциативность:
		\[\forall a, b, c \in \F: (a+b)+c = a+(b+c);\]
        3. существование нейтрального элемента по сложению (часто обозначаемого как <<0>>):
		\[\exists 0 \in \F: \forall a \in \F: a + 0 = 0 + a = a; \]
        4. существование обратного элемента:
		\[\forall a \in \F: \exists -a: a + (-a) = 0; \]
    \item выполняются аксиомы группы для операции <<умножения>>, за одним исключением: \\
        1. замкнутость:
		\[\forall a, b \in \F: a \cdot b \in \F; \]
        2. ассоциативность:
		\[\forall a, b, c \in \F: (a \cdot b) \cdot c = a \cdot (b \cdot c);\]
        3. существование нейтрального элемента по умножению (часто обозначаемого как <<1>>):
		\[\exists 1 \in \F: \forall a \in \F: a \cdot 1 = 1 \cdot a = a;\]
        4. существование обратного элемента по умножению для всех элементов множества, кроме нейтрального элемента по сложению:
		\[\forall a \in {\F \backslash 0}: \exists a^{-1}: a \cdot a^{-1} = a^{-1} \cdot a = 1;\]
    \item операции <<сложения>> и <<умножения>> коммутативны: \\
        \[ \begin{array}{l}
            \forall a, b \in \F: a + b = b + a, \\
            \forall a, b \in \F: a \cdot b = b \cdot a; \\
        \end{array} \]
    \item выполняется свойство дистрибутивности:
        \[ \forall a, b, c \in \F: a \cdot (b + c) = (a \cdot b) + (a \cdot c). \]
\end{itemize}

Примеры \emph{бесконечных} полей (с бесконечным числом элементов): поле рациональных чисел $\group{Q}$, поле вещественных чисел $\group{R}$, поле комплексных чисел $\group{C}$ с обычными операциями сложения и умножения.

В криптографии рассматриваются \emph{конечные} поля (с конечным числом элементов), называемые также \textbf{полями Галуа}.

Число элементов в любом конечном поле равно $p^n$, где $p$ -- простое\index{число!простое} число и $n$ -- натуральное число. Обозначения поля Галуа: $\GF{p}, \GF{p^n}, \F_p, \F_{p^n}$ (аббревиатура от Galois field). Все поля Галуа $\GF{p^n}$ изоморфны друг другу (существует взаимно однозначное отображение между полями, сохраняющее действие всех операций). Другими словами, существует только одно поле Галуа $\GF{p^n}$ для фиксированных $p, n$.

Приведём примеры конечных полей.

Двоичное поле $\GF{2}$ состоит из двух элементов. Однако задать его можно разными способами:
\begin{itemize}
	\item Как множество из двух чисел <<0>> и <<1>> с определёнными на нём операциями <<сложение>> и <<умножение>> как сложение и умножение чисел по модулю 2. Нейтральным элементом по сложению будет <<0>>, по умножению -- <<1>>:
\[\begin{array}{ll}
	0 + 0 = 0,	& 	0 \cdot 0 = 0, \\
	0 + 1 = 1,	& 	0 \cdot 1 = 0, \\
	1 + 0 = 1,	& 	1 \cdot 0 = 0, \\
	1 + 1 = 0,	& 	1 \cdot 1 = 1. \\
\end{array}\]
	\item Как множество из двух логических объектов <<ЛОЖЬ>> ($F$) и <<ИСТИНА>> ($T$) с определёнными на нём операциями <<сложение>> и <<умножение>> как булевые операции <<исключающее или>> и <<и>> соответственно. Нейтральным элементом по сложению будет <<ЛОЖЬ>>, по умножению -- <<ИСТИНА>>:
\[\begin{array}{ll}
	F + F = F,	& 	F \cdot F = F, \\
	F + T = T,	& 	F \cdot T = F, \\
	T + F = T,	& 	T \cdot F = F, \\
	T + T = F,	& 	T \cdot T = T. \\
\end{array}\]
	\item Как множество из двух логических объектов <<ЛОЖЬ>> ($F$) и <<ИСТИНА>> ($T$) с определёнными на нём операциями <<сложение>> и <<умножение>> как булевые операции <<эквивалентность>> и <<или>> соответственно. Нейтральным элементом по сложению будет <<ИСТИНА>>, по умножению -- <<ЛОЖЬ>>:
\[\begin{array}{ll}
	F + F = T,	& 	F \cdot F = F, \\
	F + T = F,	& 	F \cdot T = T, \\
	T + F = F,	& 	T \cdot F = T, \\
	T + T = T,	& 	T \cdot T = T. \\
\end{array}\]
	\item Как множество из двух чисел <<0>> и <<1>> с определёнными на нём операциями <<сложение>> и <<умножение>>, заданными в табличном представлении. Нейтральным элементом по сложению будет <<1>>, по умножению -- <<0>>:
\[\begin{array}{ll}
	0 + 0 = 1,	& 	0 \cdot 0 = 0, \\
	0 + 1 = 0,	& 	0 \cdot 1 = 1, \\
	1 + 0 = 0,	& 	1 \cdot 0 = 1, \\
	1 + 1 = 1,	& 	1 \cdot 1 = 1. \\
\end{array}\]
\end{itemize}

Все перечисленные выше варианты множеств изоморфны друг другу. Поэтому в дальнейшем под конечным полем $\GF{p}$, где $p$ -- простое\index{число!простое} число, будем понимать поле, заданное как множество целых чисел от $0$ до $p-1$ включительно, на котором операции <<сложение>> и <<умножение>> заданы как операции сложения и умножения чисел по модулю числа $p$. Например, поле $\GF{7}$ будем считать состоящим из 7-ми чисел $\{0, 1, 2, 3, 4, 5, 6\}$ с операциями умножения $(\cdot \mod 7)$ и сложения $(+ \mod 7)$ по модулю.

Конечное поле $\GF{p^n}, n > 1$ строится \textbf{расширением} \emph{базового} поля $\GF{p}$. Элемент поля представляется как многочлен степени $n-1$ (или меньше) с коэффициентами из базового поля $\GF{p}$:
    \[ \alpha = \sum\limits_{i=0}^{n-1} a_i x^i, ~ a_i \in \GF{p}. \]

Операция сложения элементов в таком поле традиционно задаётся как операция сложения коэффициентов при одинаковых степенях в базовом поле $\GF{p}$. Операция умножения -- как умножение многочленов со сложением и умножением коэффициентов в базовом поле $\GF{p}$ и дальнейшим приведением результата по модулю некоторого заданного (для поля) неприводимого\footnote{Многочлен называется \textbf{неприводимым}\index{многочлен!неприводимый}, если он не раскладывается на множители, и \textbf{приводимым}\index{многочлен!приводимый}, если раскладывается.} многочлена $m(x)$. Количество элементов в поле равно $p^n$.

Многочлен $g(x)$ называется \textbf{примитивным элементом}\index{многочлен!примитивный} (генератором) поля, если его степени порождают все ненулевые элементы, то есть $\GF{p^n} \setminus \{0\}$, заданное неприводимым многочленом $m(x)$, за исключением нуля:
    \[ \GF{p^n} \setminus \{0\} = \{ g(x), g^2(x), g^3(x), \dots, g^{p^n-1}(x) = 1 \mod m(x) \}. \]

\example
В таблице~\ref{tab:irreducible-gf2} приведены примеры многочленов \emph{над полем} $\GF{2}$.
\begin{table}[!ht]
    \centering
    \caption{Пример многочленов над полем $\GF{2}$\label{tab:irreducible-gf2}}
    \begin{tabular}{|c|c|c|}
        \hline
        Многочлен & \parbox{2.5cm}{Упрощённая запись} & Разложение \\
        \hline
        $'1' x + '0'$ & $x$ & неприводимый \\
        $'1' x + '1'$ & $x+1$ & неприводимый \\
        $'1' x^2 + '0' x + '0'$ & $x^2$ & $x \cdot x$ \\
        $'1' x^2 + '0'x + '1'$ & $x^2 + 1$ & $(x+1) \cdot (x+1)$ \\
        $'1' x^2 + '1' x + '0'$ & $x^2 + x$ & $x \cdot (x+1)$ \\
        $'1' x^2 + '1' x + '1'$ & $x^2 + x + 1$ & неприводимый \\
        $'1' x^3 + '0' x^2 + '0' x + '1'$ & $x^3 + 1$ & $(x+1) \cdot (x^2+x+1)$ \\
        \hline
    \end{tabular}
\end{table}
\exampleend


\section{Конечные поля и операции в алгоритме AES}\index{шифр!AES|(}
\selectlanguage{russian}

В алгоритме блочного шифрования\index{шифр!блочный} AES преобразования над байтами и битами осуществляются специальными математическими операциями. Биты и байты понимаются как элементы поля.

\subsection{Операции с байтами в AES}

Чтобы определить операции сложения и умножения двух байтов, введём сначала представление байта в виде многочлена степени 7 или меньше. Байт
    \[ a =( a_7, a_6, a_5, a_4, a_3, a_2, a_1, a_0) \]
преобразуется в многочлен $a(x)$ с коэффициентами 0 или 1 по правилу
    \[ a(x) = a_{7}x^{7}+a_{6}x^{6}+a_{5}x^{5}+a_{4}x^{4}+a_{3}x^{3}+a_{2}x^{2}+a_{1}x+a_{0}. \]

Далее байт трактуется как элемент конечного поля $\GF{2^8}$, заданного неприводимым многочленом
    \[ m(x) = x^{8}+x^{4}+x^{3}+x +1. \]

Произведение многочленов $a(x)$ и $b(x)$ по модулю многочлена $m(x)$ записывают как
    \[ c(x) = a(x) b(x) \mod m(x). \]
Остаток $c(x)$ представляет собой многочлен степени 7 или меньше. Его коэффициенты $(c_{7}, c_{6}, c_{5}, c_{4}, c_{3}, c_{2}, c_{1}, c_{0})$ образуют байт $c$, который и называется произведением байтов $a$ и $b$.

Сложение байтов осуществляется по $\oplus$ (исключающее ИЛИ), что является операцией сложения многочленов в двоичном поле.

\emph{Единичным} элементом поля является байт $\mathrm{'00000001'}$, или $\mathrm{'01'}$ в шестнадцатеричной записи. \emph{Нулевым} элементом поля является байт $\mathrm{'00000000'}$, или $\mathrm{'00'}$ в шестнадцатеричной записи. Одним из \emph{примитивных} элементов поля является байт $\mathrm{'00000010'}$, или $\mathrm{'02'}$ в шестнадцатеричной записи. Байты часто записывают в шестнадцатеричной форме, но при математических преобразованиях они должны интерпретироваться как элементы поля $\GF{2^8}$.

Для каждого ненулевого байта $a$ существует обратный байт $b$ такой, что их произведение является единичным байтом:
    \[ a b = 1 \mod m(x). \]
Обратный байт обозначается $b = a^{-1}$.

Для байта $a$, представленного многочленом $a(x)$, нахождение обратного байта $a^{-1}$ сводится к решению уравнения
    \[ m(x) d(x) + b(x) a(x) = 1. \]
Если такое решение найдено, то многочлен $b(x) \mod m(x)$ является представлением обратного байта $a^{-1}$. Обратный элемент (байт) может быть найден с помощью расширенного алгоритма Евклида для многочленов.

\example 1
Найти байт, обратный байту $a = \mathrm{'C1'}$ в шестнадцатеричной записи. Так как $a(x) = x^{7} + x^{6} + 1$, то с помощью расширенного алгоритма Евклида находим
    \[ (x^{8} + x^{4} + x^{3} + x + 1) (x^{4} + x^{3} + x^{2} + x + 1) + (x^{7} + x^{6} + 1) (x^{5} + x^{3}) = 1. \]
Таким образом, обратный элемент поля, или обратный байт $\mathrm{'C1'}$, равен
    \[ x^{5} + x^{3} = a^{-1} = \mathrm{'00101000'} = \mathrm{'28'}. \]
\exampleend

\example 2
В алгоритме блочного шифрования AES байты рассматриваются как элементы поля Галуа $\GF{2^8}$. Сложим байты $\mathrm{'57'}$ и $\mathrm{'83'}$. Представляя их многочленами, находим
    \[ (x^6 + x^4 + x^2 + x + 1) + (x^7 + x + 1) = x^7 + x^6 + x^4 + x^2, \]
или в двоичной записи --
    \[ 01010111 \oplus 10000011 = 11010100 = \mathrm{'D4'}. \]
Получили $\mathrm{'57'} + \mathrm{'83'} = \mathrm{'D4'}$.
\exampleend

\example 3
Выполним в поле $\GF{2^8}$, заданном неприводимым многочленом
    \[ m(x) = x^8 + x^4 + x^3 + x + 1 \]
(из алгоритма AES), операции с байтами: $\mathrm{'FA'} \cdot \mathrm{'A9'} + \mathrm{'E0'}$, где
    \[ FA = 11111010, ~ A9 = 10101001, ~ E0 = 11100000, \]
    \[ (x^7 + x^6 + x^5 + x^4 + x^3  +x)(x^7 + x^5 + x^3 + 1) + (x^7 + x^6 + x^5) \mod m(x) = \]
    \[ = x^{14} + x^{13} + x^{10} + x^{8} + x^7 + x^3 + x \mod m(x) = \]
    \[ = (x^{14} + x^{13} + x^{10} + x^{8} + x^7 + x^3 + x) + x^6 \cdot m(x) \mod m(x) = \]
    \[ = x^{13} + x^9 + x^8 + x^6 + x^3 + x \mod m(x) = \]
    \[ = (x^{13} + x^9 + x^8 + x^6 + x^3 + x) + x^5 \cdot m(x) \mod m(x) = \]
    \[ = x^5 + x^3 + x \mod m(x) = \mathrm{'2A'}. \]
\exampleend


\subsection{Операции над вектором из байтов в AES}
%\subsection{Многочлены над полем в алгоритме AES}

Поле $\GF{2^{nk}}$ можно задать как расширение степени $nk$ базового поля $\GF{2}$:
    \[ \alpha \in \GF{2^{nk}}, ~ \alpha = \sum\limits_{i=0}^{nk-1} a_i x^i, ~ a_i \in \GF{2} \]
с неприводимым многочленом $r(x)$ степени $nk$ над полем $\GF{2}$,
    \[ r(x) = \sum\limits_{i=0}^{nk} a_i x^i, ~ a_i \in \GF{2}, ~ a_{nk} = 1. \]

Поле $\GF{2^{nk}}$ можно задать и как расширение степени $k$ базового поля $\GF{2^n}$:
    \[ \alpha \in \GF{(2^n}^k), ~ \alpha = \sum\limits_{i=0}^{k-1} a_i x^i, ~ a_i \in \GF{2^n} \]
с неприводимым многочленом $r(x)$ степени $k$ над полем $\GF{2^n}$,
    \[ r(x) = \sum\limits_{i=0}^{k} a_i x^i, ~ a_i \in \GF{2^n}, ~ a_k = 1. \]

\example 1
В таблице~\ref{tab:irreducible-gf8} приведены примеры приводимых и неприводимых многочленов над полем $\GF{2^8}$.
\begin{table}[!ht]
    \centering
    \caption{Примеры многочленов над полем $\GF{2^8}$\label{tab:irreducible-gf8}}
    \begin{tabular}{|c|c|}
        \hline
        Многочлен & Разложение \\
        \hline
        $\mathrm{'01'} x + \mathrm{'00'}$ & неприводимый \\
        $\mathrm{'01'} x + \mathrm{'01'}$ & неприводимый \\
        $\mathrm{'01'} x + \mathrm{'A9'}$ & неприводимый \\
        $\mathrm{'01'} x^2 + \mathrm{'00'} x + \mathrm{'00'}$ & $(\mathrm{'01'} x) \cdot (\mathrm{'01'} x)$ \\
        $\mathrm{'1D'} x^2 + \mathrm{'AF'} x + \mathrm{'52'}$ & $(\mathrm{'41'} x + \mathrm{'0A'}) \cdot (\mathrm{'E3'} x + \mathrm{'5A'})$ \\
        $\mathrm{'01'} x^4 + \mathrm{'01'}$ & $(\mathrm{'01'} x + \mathrm{'01'})^4$ \\
        \hline
    \end{tabular}
\end{table}
\exampleend

В алгоритме AES вектор-столбец $\mathbf{a}$ состоит из четырёх байтов $a_{0}, a_{1}, a_{2}, a_{3}$. Ему ставится в соответствие многочлен $\mathbf{a}(y)$ от переменной $y$ вида
    \[ \mathbf{a}(y) = a_{3}y^{3}+a_{2}y^{2}+a_{1}y+a_{0}, \]
причём, коэффициенты многочлена (байты) интерпретируются как элементы поля $\GF{2^{8}}$. Это значит, что при сложении или умножении двух таких многочленов их коэффициенты складываются и перемножаются, как определено выше.

Многочлены $\mathbf{a}(y)$ и $\mathbf{b}(y)$ умножаются по модулю многочлена
    \[ \mathbf{M}(y)= \mathrm{'01'} y^4 + \mathrm{'01'} = y^4 + 1, ~ \mathrm{'01'} \in \GF{2^8}, \]
    \[ \mathbf{M}(y)= (\mathrm{'01'}, \mathrm{'00'},\mathrm{'00'}, \mathrm{'00'}, \mathrm{'01'}), \]
который \emph{не} является неприводимым над $\GF{2^8}$.
%Следовательно, многочлен $\mathbf{a}(y)$ задаёт многочлен третьей степени над полем $\GF{2^8}$, но не является элементом поля $\GF{2^{32}}$.

Операция умножения по модулю $\mathbf{M}(y)$ обозначается $\otimes$:
    \[ \mathbf{a}(y) ~ \mathbf{b}(y) \mod \mathbf{M}(y) ~\equiv~ \mathbf{a}(y) \otimes \mathbf{b}(y). \]

Операция <<перемешивание столбца>> в шифровании AES состоит в умножении многочлена столбца на
    \[ \mathbf{c}(y) = (03, 01, 01, 02) = \mathrm{'03'} y^3 + \mathrm{'01'} y^2 + \mathrm{'01'} y + \mathrm{'02'} \]
по модулю $\mathbf{M}(y)$. Многочлен $\mathbf{c}(y)$ имеет обратный многочлен
    \[ \mathbf{d}(y) = \mathbf{c}^{-1}(y) \mod \mathbf{M}(y) = (\mathrm{0B}, \mathrm{0D}, \mathrm{09}, \mathrm{0E}) = \]
        \[ = \mathrm{'0B'} y^3 + \mathrm{'0D'} y^2 + \mathrm{'09'} y + \mathrm{'0E'}, \]
    \[ \mathbf{c}(y) \otimes \mathbf{d}(y) = (00, 00, 00, 01) = 1. \]
При расшифровании выполняется умножение на $\mathbf{d}(y)$ вместо $\mathbf{c}(y)$.

Так как
    \[ y^j = y^{j \mod 4} \mod y^4+1, \]
где коэффициенты из поля $\GF{2^8}$, то произведение многочленов
    \[ \mathbf{a}(y) = a_{3}y^{3}+ a_{2}y^{2} + a_{1}y + a_{0} \]
и
    \[ \mathbf{b}(y) = b_{3}y^{3} + b_{2}y^{2} + b_{1}y + b_{0}, \]
обозначаемое как
    \[ \mathbf{f}(y) = \mathbf{a}(y) \otimes \mathbf{b}(y) = f_{3}y^{3} + f_{2}y^{2} + f_{1}y + f_{0}, \]
содержит коэффициенты
\[
    \begin{array}{ccccccccc}
        f_{0} & = & a_{0}b_{0} & + & a_{3}b_{1} & + & a_{2}b_{2} & + & a_{1}b_{3}, \\
        f_{1} & = & a_{1}b_{0} & + & a_{0}b_{1} & + & a_{3}b_{2} & + & a_{2}b_{3}, \\
        f_{2} & = & a_{2}b_{0} & + & a_{1}b_{1} & + & a_{0}b_{2} & + & a_{3}b_{3}, \\
        f_{3} & = & a_{3}b_{0} & + & a_{2}b_{1} & + & a_{1}b_{2} & + &  a_{0}b_{3}.
    \end{array}
\]

Эти соотношения можно переписать также в матричном виде:
\[
    \begin{array}{cccc}
        \left[ \begin{array}{c}
            f_{0} \\
            f_{1} \\
            f_{2} \\
            f_{3}
        \end{array} \right] &  = & \left[\begin{array}{cccc}
            a_{0} & a_{3} & a_{2} & a_{1} \\
            a_{1} & a_{0} & a_{3} & a_{2} \\
            a_{2} & a_{1} & a_{0} & a_{3} \\
            a_{3} & a_{2} & a_{1} & a_{0}
        \end{array}\right] & \left[\begin{array}{c}
            b_{0} \\
            b_{1} \\
            b_{2} \\
            b_{3}
        \end{array} \right]
    \end{array}.
\]

Умножение матриц производится в поле $\GF{2^8}$. Матричное представление полезно, если нужно умножать фиксированный вектор на несколько различных векторов.

\example 2
Вычислим для $\mathbf{a}(y) = (\mathrm{F2}, \mathrm{7E}, \mathrm{41}, \mathrm{0A})$ произведение $\mathbf{a}(y) \otimes \mathbf{c}(y)$:
\[
    \mathbf{c}(y) = (03, 01, 01, 02),
\] \[
    \mathbf{d}(y) = \mathbf{c}^{-1}(y) \mod \mathbf{M}(y) = (\mathrm{0B}, \mathrm{0D}, \mathrm{09}, \mathrm{0E}).
\] \[
    \mathbf{a}(y) \otimes \mathbf{c}(y) =
    \left[ \begin{array}{cccc}
        \mathrm{0A} & \mathrm{F2} & \mathrm{7E} & \mathrm{41} \\
        \mathrm{41} & \mathrm{0A} & \mathrm{F2} & \mathrm{7E} \\
        \mathrm{7E} & \mathrm{41} & \mathrm{0A} & \mathrm{F2} \\
        \mathrm{F2} & \mathrm{7E} & \mathrm{41} & \mathrm{0A} \\
    \end{array} \right] \cdot
    \left[ \begin{array}{c} \mathrm{02} \\ \mathrm{01} \\ \mathrm{01} \\ \mathrm{03} \end{array} \right] =
\] \[
    \left[ \begin{array}{ccccccc}
        \mathrm{0A} \cdot \mathrm{02} & \oplus & \mathrm{F2} & \oplus & \mathrm{7E} & \oplus & \mathrm{41} \cdot \mathrm{03} \\
        \mathrm{41} \cdot \mathrm{02} & \oplus & \mathrm{0A} & \oplus & \mathrm{F2} & \oplus & \mathrm{7E} \cdot \mathrm{03} \\
        \mathrm{7E} \cdot \mathrm{02} & \oplus & \mathrm{41} & \oplus & \mathrm{0A} & \oplus & \mathrm{F2} \cdot \mathrm{03} \\
        \mathrm{F2} \cdot \mathrm{02} & \oplus & \mathrm{7E} & \oplus & \mathrm{41} & \oplus & \mathrm{0A} \cdot \mathrm{03} \\
    \end{array} \right] =
    \left[ \begin{array}{c} \mathrm{5B} \\ \mathrm{F8} \\ \mathrm{BA} \\ \mathrm{DE} \end{array} \right];
\] \[
    \begin{array}{l}
        \mathbf{a}(y) \otimes \mathbf{c}(y) = \mathbf{b}(y), \\
        \mathbf{b}(y) \otimes \mathbf{d}(y) = \mathbf{a}(y); \\
    \end{array}
\] \[
    \begin{array}{ccccc}
        (\mathrm{F2}, \mathrm{7E}, \mathrm{41}, \mathrm{0A}) & \otimes & (\mathrm{03}, \mathrm{01}, \mathrm{01}, \mathrm{02}) & = & (\mathrm{DE}, \mathrm{BA}, \mathrm{F8}, \mathrm{5B}), \\
        (\mathrm{DE}, \mathrm{BA}, \mathrm{F8}, \mathrm{5B}) & \otimes & (\mathrm{0B}, \mathrm{0D}, \mathrm{09}, \mathrm{0E}) & = & (\mathrm{F2}, \mathrm{7E}, \mathrm{41}, \mathrm{0A}). \\
    \end{array}
\]
\exampleend

\index{шифр!AES|)}


\input{modular_ariphmetics}

\input{pseudo-primes}

\input{groups_of_ec_points_over_finite_fields}

\section[Полиномиальные и экспоненциальные алгоритмы]{Полиномиальные и \\ экспоненциальные алгоритмы}

Данный раздел поясняет обоснованность стойкости криптосистем с открытым ключом и имеет лишь косвенное отношение к дискретной математике.

Машина Тьюринга (МТ) (модель, представляющая любой вычислительный алгоритм) состоит из следующих частей:
\begin{itemize}
    \item неограниченная лента, разделённая на клетки; в каждой клетке содержится символ из конечного алфавита, содержащего пустой символ blank; если символ ранее не был записан на ленту, то он считается blank;
    \item печатающая головка, которая может считать, записать символ $a_i$ и передвинуть ленту на 1 клетку влево-вправо $d_k$;
    \item конечная таблица действий
    \[ (q_i, a_j) \rightarrow (q_{i1}, a_{j1}, d_k), \]
где $q$ -- состояние машины.
\end{itemize}

Если таблица переходов однозначна, то машина Тьюринга\index{машина Тьюринга} называется детерминированной. \textbf{Детерминированная} машина Тьюринга может \emph{имитировать} любую существующую детерминированную ЭВМ. Если таблица переходов неоднозначна, то есть $(q_i, a_j)$ может переходить по нескольким правилам, то машина \textbf{недетерминированная}. \emph{Квантовый компьютер} является примером недетерминированной машины Тьюринга.

Класс задач $\set{P}$ -- задачи, которые могут быть решены за \emph{полиномиальное} время\index{задача!полиномиальная} на \emph{детерминированной} машине Тьюринга. Пример полиномиальной сложности (количество битовых операций)
    \[ O(k^{\textrm{const}}), \]
где $k$ -- длина входных параметров алгоритма. Операция возведения в степень в модульной арифметике $a^b \mod n$ имеет кубическую сложность $O(k^3)$, где $k$ -- двоичная длина чисел $a,b,n$.

Класс задач $\set{NP}$ -- обобщение класса $\set{P} \subseteq \set{NP}$, включает задачи, которые могут быть решены за \emph{полиномиальное} время на \emph{недетерминированной} машине Тьюринга. Пример сложности задач из $\set{NP}$ -- экспоненциальная сложность\index{задача!экспоненциальная}
    \[ O(\textrm{const}^k). \]
Описанный в разделе криптостойкости системы Эль-Гамаля\index{криптосистема!Эль-Гамаля} алгоритм Гельфонда решения задачи дискретного логарифмирования (нахождения $x$ для заданных основания $g$, модуля $p$ и $a = g^x \mod p$) имеет сложность $O(e^{k/2})$, где $k$ -- двоичная длина чисел.

В криптографии полиномиальные $\set{P}$ алгоритмы считаются \emph{лёгкими и вычислимыми} на ЭВМ, которые являются детерминированными машинами Тьюринга. Неполиномиальные (экспоненциальные) $\set{NP}$ алгоритмы считаются \emph{трудными и невычислимыми} на ЭВМ, так как из-за экспоненциального роста сложности всегда можно выбрать такой параметр $k$, что время вычисления станет сравнимым с возрастом Вселенной.


Класс $\set{NP}$-полных задач -- подмножество задач из $\set{NP}$, для которых не известен полиномиальный алгоритм для детерминированной машины Тьюринга, и все задачи могут быть сведены друг к другу за полиномиальное время на \emph{детерминированной} машине Тьюринга. Например, задача об укладке рюкзака является $\set{NP}$-полной.

Стойкость криптосистем с \emph{открытым} ключом, как правило, основана на $\set{NP}$ или $\set{NP}$-полных задачах:
\begin{enumerate}
    \item RSA\index{криптосистема!RSA} -- $\set{NP}$-задача факторизации (строго говоря, основана на трудности извлечения корня степени $e$ по модулю $n$).
    \item Криптосистемы типа Эль-Гамаля\index{криптосистема!Эль-Гамаля} -- $\set{NP}$-задача дискретного логарифмирования.
\end{enumerate}

\emph{Нерешённой} проблемой является доказательство неравенства
    \[ \set{P} \neq \set{NP}. \]
Именно на гипотезе о том, что для некоторых задач не существует полиномиальных алгоритмов, и основана стойкость криптосистем с открытым ключом.

\input{coincide-index_method}

\input{tasks}

\printindex

\chapter*{Литература}
\addcontentsline{toc}{chapter}{Литература}
\begingroup
\renewcommand{\chapter}[2]{}%
%\bibliographystyle{ugost2008s}
%\bibliography{bibliography}
\printbibliography
\endgroup

\end{document}
